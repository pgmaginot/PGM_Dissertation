%%%%%%%%%%%%%%%%%%%%%%%%%%%%%%%%%%%%%%%%%%%%%%%%%%%
%
%  New template code for TAMU Theses and Dissertations starting Fall 2012.  
%  For more info about this template or the 
%  TAMU LaTeX User's Group, see http://www.howdy.me/.
%
%  Author: Wendy Lynn Turner 
%	 Version 1.0 
%  Last updated 8/5/2012
%
%%%%%%%%%%%%%%%%%%%%%%%%%%%%%%%%%%%%%%%%%%%%%%%%%%%
%%%%%%%%%%%%%%%%%%%%%%%%%%%%%%%%%%%%%%%%%%%%%%%%%%%%%%%%%%%%%%%%%%%%%
%%                           ABSTRACT 
%%%%%%%%%%%%%%%%%%%%%%%%%%%%%%%%%%%%%%%%%%%%%%%%%%%%%%%%%%%%%%%%%%%%%

\chapter*{ABSTRACT}
\addcontentsline{toc}{chapter}{ABSTRACT} % Needs to be set to part, so the TOC doesnt add 'CHAPTER ' prefix in the TOC.

\pagestyle{plain} % No headers, just page numbers
\pagenumbering{roman} % Roman numerals
\setcounter{page}{2}

\indent The linear discontinuous finite element method (LDFEM) is the current work horse of the radiation transport community.  
The popularity of LDFEM is a result of LDFEM (and its $Q^1$ multi-dimensional extensions) being both accurate and preserving the thick diffusion limit.
In practice, the LDFEM equations must be ``lumped'' to mitigate negative radiation transport solutions.
Negative solutions are non-physical, but are inherent to the mathematics of LDFEM and other spatial discretizations.

Ongoing changes in high performance computing (HPC) are dictating a preference for increased numbers of floating point operations (FLOPS) per unknown.
Higher order discontinuous finite element methods (DFEM), those with polynomial trial spaces greater than linear, have been found to offer more accuracy per unknown than LDFEM.
However, DFEM with higher degree trial spaces have received only limited attention due to their increased computational time per unknown, LDFEM's preservation of the thick diffusion limit, and the relative accuracy of LDFEM compared to other historical spatial discretizations.
As solution methods evolve to make the most efficient use of HPC, it is possible that the increased computational work of higher order DFEM may become a strength rather than a hindrance.

For higher order DFEM to be useful in practice, lumping techniques must be developed to inhibit negative radiation transport solutions.
We will show that traditional mass matrix lumping does not guarantee positive solutions and limits the overall accuracy of the DFEM scheme.
To solve this problem, we propose a new, quadrature based, self-lumping technique.
Our self-lumping technique does not limit solution order of convergence, improves solution positivity, and can be easily adapted to account for the within cell variation of interaction cross section.
To test and demonstrate the characteristics of our self-lumping methodology, we apply our schemes to several test problems: a homogeneous, source-free pure absorber; a pure absorber with spatially varying cross section; a model fuel depletion problem; and finally, we solve the grey thermal radiative transfer equations.



\pagebreak{}
