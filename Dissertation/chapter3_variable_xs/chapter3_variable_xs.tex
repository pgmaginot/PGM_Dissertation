%%%%%%%%%%%%%%%%%%%%%%%%%%%%%%%%%%%%%%%%%%%%%%%%%%%
%
%  New template code for TAMU Theses and Dissertations starting Fall 2012.  
%  For more info about this template or the 
%  TAMU LaTeX User's Group, see http://www.howdy.me/.
%
%  Author: Wendy Lynn Turner 
%	 Version 1.0 
%  Last updated 8/5/2012
%
%%%%%%%%%%%%%%%%%%%%%%%%%%%%%%%%%%%%%%%%%%%%%%%%%%%
%%%%%%%%%%%%%%%%%%%%%%%%%%%%%%%%%%%%%%%%%%%%%%%%%%%%%%%%%%%%%%%%%%%%%%
%%                           SECTION III
%%%%%%%%%%%%%%%%%%%%%%%%%%%%%%%%%%%%%%%%%%%%%%%%%%%%%%%%%%%%%%%%%%%%%



\chapter{\uppercase{Last Chapter: The Importance of Research}}
\label{sec:chapter3_variable_xs}

Thermal radiative transfer interaction opacities can be rapidly varying functions of temperature.  
For example, consider Marshak wave problems and the canonical $T^{-3}$ dependence \cite{ober_shadid} of absorption opacity.
Opacity variations of several orders of magnitude near the heated/cold material interface are easily possible.
Historically, the neutron transport and thermal radiative transfer communities assumed interaction cross section and opacities, respectively, were cell-wise constant \cite{adams,lewis_book,morel_radtran}.
Adams first described \cite{adams_scb} and then presented computational results \cite{adams_nowak} for a ``simple'' corner balance (SCB) spatial discretization method that explicitly accounted for the spatial variation of opacity within individual spatial cells.
The SCB scheme (which can be shown to be related to a LDFEM for certain geometries) accounts for opacity spatial variation within each cell via vertex-based quadrature evaluation.  
Similar strategies have been adapted to LDFEM radiative diffusion \cite{ober_shadid} and LDFEM TRT \cite{warsa_lmfga} calculations.
For accurate TRT solutions, use of higher order DFEM will requires the development of corresponding higher order strategies  for treating the within cell spatial variation of opacities.

\section{New Section}

%%%%%%%%%%%%%%%%%%%%%%%%%%%%%%%%%%%%%%%%%%%%%%%%%%%%%%
\begin{figure}[H]
\centering
\includegraphics[scale=.50]{figures/Penguins.jpg}
\caption{TAMU figure}
\label{fig:tamu-fig3}
\end{figure}
%%%%%%%%%%%%%%%%%%%%%%%%%%%%%%%%%%%%%%%%%%%%%%%%%%%%%%
\section{Another Section}

Text between the figures.  Text between the figures. Text between the figures. Text between the figures.  Text between the figures. Text between the figures. Text between the figures.  Text between the figures. Text between the figures. Text between the figures.  Text between the figures. Text between the figures.
%%%%%%%%%%%%%%%%%%%%%%%%%%%%%%%%%%%%%%%%%%%%%%%%%%%%%%%
%\begin{figure}[H]
%\centering
%\includegraphics[scale=.50]{figures/Penguins.jpg}
%\caption{Another TAMU figure}
%\label{fig:tamu-fig4}
%\end{figure}
%%%%%%%%%%%%%%%%%%%%%%%%%%%%%%%%%%%%%%%%%%%%%%%%%%%%%%%

\subsection{Subsection}

%%%%%%%%%%%%%%%%%%%%%%%%%%%%%%%%%%%%%%%%%%%%%%%%%%%%%%%
%\begin{figure}[H]
%\centering
%\includegraphics[scale=.50]{figures/Penguins.jpg}
%\caption{Another TAMU figure}
%\label{fig:tamu-fig4-2}
%\end{figure}
%%%%%%%%%%%%%%%%%%%%%%%%%%%%%%%%%%%%%%%%%%%%%%%%%%%%%%%
\subsection{Subsection}

A table example is going to follow.

\begin{table}[H]
\centering
\caption{This is a table template}
\begin{tabular}{|l|c|c|c|c|c|}
\hline
Product & 1 & 2 & 3 & 4 & 5\\
\hline
Price & 124.- & 136.- & 85.- & 156.- & 23.-\\
Guarantee [years] & 1 & 2 & - & 3 & 1\\
Rating & 89\% & 84\% & 51\% & & 45\%\\
\hline
\hline
Recommended & yes & yes & no & no & no\\
\hline
\end{tabular}
\label{tab:template2}
\end{table}
\subsubsection{This is a subsubsection}
\section{Another Section}