%%%%%%%%%%%%%%%%%%%%%%%%%%%%%%%%%%%%%%%%%%%%%%%%%%%
%
%  New template code for TAMU Theses and Dissertations starting Fall 2012.  
%  For more info about this template or the 
%  TAMU LaTeX User's Group, see http://www.howdy.me/.
%
%  Author: Wendy Lynn Turner 
%	 Version 1.0 
%  Last updated 8/5/2012
%
%%%%%%%%%%%%%%%%%%%%%%%%%%%%%%%%%%%%%%%%%%%%%%%%%%%
%%%%%%%%%%%%%%%%%%%%%%%%%%%%%%%%%%%%%%%%%%%%%%%%%%%%%%%%%%%%%%%%%%%%%%
%%                           SECTION III
%%%%%%%%%%%%%%%%%%%%%%%%%%%%%%%%%%%%%%%%%%%%%%%%%%%%%%%%%%%%%%%%%%%%%



\chapter{\uppercase{Summary and Conclusions}}
\label{sec:chapter7_conclusions}

\section{Summary}

In this dissertation we have developed a new family of interpolatory DFEM spatial discretizations for discrete ordinates radiation transport.
This set of DFEM schemes is unique in that it is not limited to equally-spaced DFEM interpolation points, automatically generates diagonal mass matrices, can be used with DFEM trial spaces of arbitrary degree, and can explicitly account for the within cell variation of material properties such as interaction cross section.
We have tested this new family of DFEM techniques in a variety of slab geometry radiation transport applications including steady-state neutron transport, criticality, and time dependent thermal radiative transfer simulations.

Additionally, we have demonstrated that some canonical methods of discrete ordinates radiation transport do not always behave as expected.
First, we showed that the traditional method of mass matrix lumping limits solution accuracy for higher degree polynomial trial spaces and is robust with a linear DFEM trial space.
Later, we demonstrated that the usual assumption of a cell-wise constant interaction cross section has several negative effects in problems with spatially varying interaction cross sections.
Across all application areas considered, the assumption of a cell-wise constant cross section for problems that had cross section variation within individual mesh cells resulted in a fundamental limit on spatial order of convergence, and generated non-smooth interaction rates.
The non-smooth neutron transport interaction rates manifested themselves in our thermal radiative transfer results as a material temperature solution that contained large, non-monotonic discontinuities.
Mesh refinement can reduce the severity of the thermal radiative transfer temperature solution discontinuities, but cannot eliminate the discontinuities.
Not only are the discontinuities non-physical and a sign of limited spatial order of convergence, they also complicate and can inhibit the non-linear temperature iteration required to solve the thermal radiative transfer equations. 

\section{Conclusions}

There are two main conclusions to be made from this dissertation.
\begin{enumerate}
\item Self-lumping DFEM schemes using Gauss-Legendre (SL Gauss) or Gauss-Lobatto-Legendre (SL Lobatto) quadrature as the DFEM interpolation points are well suited to discrete ordinates radiation transport calculations.
\item Self-lumping schemes are easily modified to explicitly account for the within cell variation of material properties, resulting in methods that are significantly more accurate for problems with spatially varying material properties than those that assume cell-wise constant material properties.
\end{enumerate}

In neutron transport, criticality, and thermal radiative transfer simulations, both SL Gauss and SL Lobatto converge the $L^2$ norm of the angular flux error $\propto P+1$ for problems with cell-wise constant cross section.
SL Lobatto is robust for all odd degree DFEM trial spaces, and SL Gauss is robust for all even degree trial spaces, assuming cell-wise constant cross section.
Further, SL Lobatto is equivalent to traditional lumping for linear DFEM, but unlike traditional lumping DFEM schemes, SL Lobatto increases in spatial order of convergence with increased trial space degree.

Self-lumping DFEM schemes easily account for the variation of interaction cross or other material properties.
A $P$ degree self-lumping scheme using Gauss or Lobatto quadrature only requires $P+1$ material property evaluations to obtain schemes that converge the $L^2$ norm of the angular flux or radiation intensity $\propto P+1$.
This is in contrast to DFEM schemes that assume a cell-wise constant cross yielding only second order spatial convergence, regardless of trial space degree, for problems with spatially varying interaction cross sections.
In our neutron transport test problems, SLXS Lobatto and SLXS Gauss (the variants of SL Lobatto and SL Gauss that explicitly account for within cell variation of material properties), converged interaction rate and interaction rate dependent quantities $\propto P+1$.
However, in our thermal radiative transfer MMS testing SLXS Lobatto converged the $L^2$ error of temperature, a quantity driven by an interaction rate, $\propto P$.
Though not $P+1$ as we had hoped, given the non-linear nature of the thermal radiative transfer equations, and necessity to integrate much higher order polynomials, e.g. the Planckian term that is a $P^4$ degree polynomial, order $P$ convergence is still promising, as it still allows for increased accuracy with increasing DFEM trial space degree.
More surprising is that SLXS Gauss applied to the grey TRT appears to converge the $L^2$ error of temperature $\propto P+2$.
It should be noted though that the orders of convergence we have given here are effectively only experimental observations, and there was some disagreement between the apparent orders of convergence for certain error quantities between test problems.

\subsection{Future Work}

There are several exciting avenues for continued study and advancement if the topics and methods covered in this dissertation.
Clearly the extension of this slab geometry work to multiple spatial dimensions is required for problems of greater scientific and engineering interest and complexity.
Additionally, MIP DSA appeared to be an effective iterative acceleration technique for the grey TRT equations, and as such we would like to see how it performs as the diffusion operator for linear multi-frequency grey acceleration of the multi-group/multi-frequency thermal radiative transfer equations.

Topics of research beyond simply extending our methodology are abundant as well.
A non-exhaustive list includes:
\begin{enumerate}
\item developing a theory to explain and apriori predict whether a given matrix lumping technique will yield a robust solution,
\item explaining the apparent super convergence of SLXS Gauss for the TRT temperature solution, 
\item conducting a diffusion limit analysis of higher order trial space DFEM, and
\item developing additional TRT manufactured solutions that challenge spatial discretization more completely and are closer to real-world applications in both nature and scaling.
\end{enumerate}
