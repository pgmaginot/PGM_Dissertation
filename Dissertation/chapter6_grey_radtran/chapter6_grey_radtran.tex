%%%%%%%%%%%%%%%%%%%%%%%%%%%%%%%%%%%%%%%%%%%%%%%%%%%
%
%  New template code for TAMU Theses and Dissertations starting Fall 2012.  
%  For more info about this template or the 
%  TAMU LaTeX User's Group, see http://www.howdy.me/.
%
%  Author: Wendy Lynn Turner 
%	 Version 1.0 
%  Last updated 8/5/2012
%
%%%%%%%%%%%%%%%%%%%%%%%%%%%%%%%%%%%%%%%%%%%%%%%%%%%
%%%%%%%%%%%%%%%%%%%%%%%%%%%%%%%%%%%%%%%%%%%%%%%%%%%%%%%%%%%%%%%%%%%%%%
%%                           SECTION III
%%%%%%%%%%%%%%%%%%%%%%%%%%%%%%%%%%%%%%%%%%%%%%%%%%%%%%%%%%%%%%%%%%%%%



\chapter{\uppercase{Grey Thermal Radiative Transfer- Numerical Results}}
\label{sec:chapter6_grey_radtran_results}


We now consider several test problems to verify and demonstrate the capability of the self-lumping DFEM schemes we have developed to solve the grey TRT equations.
First, in \secref{sec:chap6_analytic_results} we will consider problems with analytic solutions to verify and demonstrate the asymptotic convergence rates of our DFEM and SDIRK discretizations of the grey thermal radiative transfer equations.
In \secref{sec:marshak_waves}, we will consider two Marshak wave problems.  
We consider a Marshak wave test problem to be an initially cold slab being heated by a strong, incident photon source,  creating a nearly discontinuous solution as the thermal wave propagates through and heats the cold slab.
Finally, we will summarize the effectiveness of the modified interior penalty diffusion synthetic acceleration operator in accelerating the iterative convergence of the grey TRT equations in \secref{sec:mip_results}.


\section{Problems with Analytic Solutions}
\label{sec:chap6_analytic_results}


\subsection{Su-Olson Problem}

The Su-Olson problem \cite{su_olson_1} is an analytic benchmark that consists of an initially cold (initial radiation energy density and temperature conditions are identically absolute zero) half-space slab, heated for a finite amount of time by a volumetric, isotropic radiation source.
The slab's scattering opacity is constant in space and temperature, absorption opacity is constant in space and temperature, and $C_v \propto T^3$.
Assuming $C_v = \alpha T^3$ is critical; as Long, et al. \cite{alex_paper} noted, the assumption regarding $C_v$ is not physical, but is required to make the thermal radiative transfer equations linear in $I$ and $T^4$, or conversely linear in $I$ and material energy density. 
After a series of transformations, Su and Olson derived an analytic solution to the thermal radiative transfer equations under these conditions; their solution is more accurately described as being semi-analytic.
While the radiation energy density and material temperature at every point can be expressed as a closed form integral, evaluation of each integral requires numerical estimation.
Further, the integral is a 2-D, indefinite integral (in both variables) of a trigonometric function with a slowly  decaying exponential argument.
However, \cite{su_olson_1} provides several radiation energy density and material energy density points in space, and thus the Su-Olson problem is beneficial as a benchmark problem to verify the physics of a given radiative transfer implementation.

Given the initial temperature condition is explicitly zero, this implies the initial $C_v$ is also zero.
This is problematic when solving explicitly for temperature and not material energy.
A near-zero heat capacity would result in the material rapidly heating, but a zero heat capacity implies a material that cannot accept heat, and thus can never be heated up.
To prevent this problem, we modify the definition of $C_v$:
\benum
C_v = 10^{-8} + \alpha T^3 \pep
\eenum
Alternatively, we could set an initial temperature to be a non absolute zero value.

Since \cite{su_olson_1} presents results in a non-dimensional format, we elect to define $a=c=1$, $\sigma_a = 1$, $\sigma_s=0$, $\alpha = 4$, truncate the full half space to $x\in[0,5]$, and define the reference temperature, $T_H = 1$.
We solve the problem using 200 spatial mesh cells, linear SLXS Lobatto DFEM, the Alexander 2-2 time differencing scheme, an initial time step size of $\Delta t = 10^{-5}$, and increase the time step size by a factor of 1.1 until a maximum time step size of $\Delta t = 10^{-3}$ is reached.

In \fig{fig:su_olson_s2_rad} we present the radiation energy density solution, $W(x)$ (in the notation on \cite{su_olson_1}), for $S_2$ angular differencing plotted against the analytic diffusion and transport solutions.  Likewise, in \fig{fig:su_olson_s2_mat}, we plot the material energy density, $V(x)$ (in the notation on \cite{su_olson_1}) for $S_2$ angular differencing.
Solutions at various times, $\tau = 1$ and $\tau=10$ are given in both plots.
As expected, the $S_2$ solution is nearly identical to diffusion, but skews slightly in the direction of the full transport solution.
\begin{figure}[!htp]
\centering
\includegraphics[width=8cm,trim=1.75in  0.5in 0.75in 0.5in,clip=true]{chapter6_grey_radtran/Dissertation_Data/Su_Olson_S2_Radiation_Energy.pdf}
\caption{$S_2$ radiation energy density profile for Su-Olson problem.}
\label{fig:su_olson_s2_rad}
\end{figure}
\begin{figure}[!hbp]
\centering
\includegraphics[width=8cm,trim=1.75in  0.5in 0.75in 0.5in,clip=true]{chapter6_grey_radtran/Dissertation_Data/Su_Olson_S2_Material_Energy.pdf}
\caption{$S_2$ material energy density profile for Su-Olson problem.}
\label{fig:su_olson_s2_mat}
\end{figure}

Increasing the number of discrete ordinates, the radiation energy density profile and material energy density profiles are given in \fig{fig:su_olson_s8_rad} and \fig{fig:su_olson_s8_mat} for $S_8$ angular differencing.
\begin{figure}[!htp]
\centering
\includegraphics[width=8cm,trim=1.75in  0.5in 0.75in 0.5in,clip=true]{chapter6_grey_radtran/Dissertation_Data/Su_Olson_S8_Radiation_Energy.pdf}
\caption{$S_8$ radiation energy density profile for Su-Olson problem.}
\label{fig:su_olson_s8_rad}
\end{figure}
\begin{figure}[!hbp]
\centering
\includegraphics[width=8cm,trim=1.75in  0.5in 0.75in 0.5in,clip=true]{chapter6_grey_radtran/Dissertation_Data/Su_Olson_S8_Material_Energy.pdf}
\caption{$S_8$ material energy density profile for Su-Olson problem.}
\label{fig:su_olson_s8_mat}
\end{figure}
By adding a few discrete directions to the quadrature set, our numerical solution becomes indistinguishable from published results of \cite{su_olson_1}, and we conclude that our TRT equation implementation is valid.

\newpage

\subsection{Method of Manufactured Solutions}

The method of manufactured solutions, MMS, consists of choosing an analytic solution, then defining the driving source necessary to achieve that solution \cite{mms}.
Though MMS solutions are derived, they are excellent for testing the convergence of numerical methods devised to solve complex physics phenomena.
While analytic solutions that do not use exotic source terms are more appealing for verification purposes, these types of solutions are difficult to devise for complex physics phenomena, especially for coupled physical phenomena like radiative transfer.  
For those multiphysics problems for which an analytic solution does exist, the solutions are usually semi-analytic, and cannot be used effectively as a reference solution for a convergence study because the semi-analytic solution has numerical errors of the same magnitude (or greater) than the numerical simulations we are attempting to verify.
%Similarly experimental results will contain measurement errors, and will almost certainly be of low resolution, either measuring only integral quantities, or providing data at only a few points in space, and thus do not permit convergence studies.

We elect to choose separable MMS of the form:
\begin{subequations}
\beanum
I_d(x,\mu_d,t) &=& M(\mu_d) F(t) W_I(x) \\
T(x) &=& F(t) W_T(x) \\
\phi(x) &=& C_M F(t) W_I(x) \\
C_M &=& \sum_{d=1}^{N_{dir}} { w_d M(\mu_d) } \pep
\eeanum
\label{eq:mms_solns}
\end{subequations}
where $M(\mu_d)$ is as angular component of the intensity solution desired, $I_d(x,\mu_d,t)$; $F(t)$ is the time component of the MMS, chosen to be the same for intensity, temperature, and angle integrated intensity; $W_I(x)$ is the spatial component of the intensity, and $W_T(x)$ is the spatial component of the temperature solutions.
$W_I(x)$ being the spatial component of $\phi$, and the definition of $C_M$ given in \eqts{eq:mms_solns} is required for the $S_N$ approximation to be true.

\subsubsection{Linear in Time - Trigonometric in Space  }
In our first method of manufactured solutions problem, MMS1, we attempt to reproduce a TRT problem whose radiation and temperature solutions is a cosine in space and  linear in time.
We use a non-dimensional form of the TRT, assuming that $a=c=1$, and for this problem we assume opacities are constant functions of space and temperature.
We impose the solution:
\beanum
M(\mu_d) &=& \frac{1}{4\pi} \\
W_I(x) &=& 10 \cos\left( \frac{\pi x}{10} - \frac{\pi}{2} \right) + 15\\
W_T(x) &=&  25 \cos\left( \frac{\pi x}{10} - \frac{\pi}{2} \right) + 30\\
F(t) &=& 1+.02t \pep
\eeanum
Physically, $x\in[0,10]$, $C_v = 0.1$, $\sigma_a = 100.0$, $\sigma_s = 0.5$, we start the simulation at $t_0 = 0$, end the simulation at $t_{end}=1$, 
taking 100 equal time steps of length $\Delta t = 0.01$, with Alexander's two stage, second order SDIRK method, which we refer to as 2-2 \cite{alexander}.  
For this and all other MMS simulations, our initial conditions are
\beanum
I(x,\mu_d,t_0) &=& M(\mu_d)W_I(x)F(t_0) \\
T(x,t_0) &=& W_T(x)F(t_0) \\
\phi(x,t_0) &=& \left[ \sum_{d=1}^{N_{dir}}{w_d M(\mu_d)} \right] W_I(x)F(t_0) \pep
\eeanum
Likewise, we always use an incident flux intensity boundary condition, for $\mu_d >0$:
\benum
I_{in}(\mu_d,t) = M(\mu_d) W_I(x_{1/2}) F(t) \pec
\eenum
and for $\mu_d < 0$
\benum
I_{in}(\mu_d,t) = M(\mu_d) W_I(x_{N_{cell}+1/2}) F(t) \pec
\eenum
where $x_{1/2}$ is the left domain boundary and $x_{N_{cell}+1/2}$ is the right domain boundary.
We do not impose, and no boundary conditions are required for the material temperature.

The Butcher tableaux for this and all other SDIRK schemes considered is given in \secref{sec:time_convergence}.
We evaluate $\vec{S}_I$ and $\vec{S}_T$ using the spatially analytic definitions of $S_I$ and $S_T$ respectively, and numerically integrate the source moments using a $N_q$ Gauss-Legendre quadrature points, with $N_q = 2N_P + 5$, and $N_P$ is the number of DFEM interpolation points, as defined previously.

We consider three DFEM schemes,
\begin{enumerate}
\item TL: traditional mass matrix lumping using equally-spaced interpolation points,
\item SL Lobatto: self-lumping quadrature using Lobatto interpolation points,
\item SL Gauss: self-lumping quadrature using Gauss interpolation points, and
\end{enumerate}  
compare each method's accuracy in calculating the discrete $L_2$ norm of the error in angle integrated intensity, $E_{\phi} = \norm{ \widetilde{\phi}(x) - \phi(x)}_{L^2}$, and the $L_2$ error in temperature, $E_T = \norm{ \widetilde{T}(x) - T(x) }_{L^2}$.
We calculate $E_{\phi}$ as
\benum
E_{\phi} = \sqrt{\sum_{c=1}^{N_{cell}}{ \frac{\Delta x_c}{2} \sum_{q=1}^{N_{qf}}{ w_q \left(\widetilde{\phi}(s_q , t_{end}) - \phi(s_q,t_{end})  \right)^2 } } }\pec
\label{eq:e_phi_6}
\eenum
where $w_q$, $s_q$ are Gauss quadrature points and $N_{qf} = 2P+7$.  $E_{T}$ is calculated analogously to $E_{\phi}$.
Figure \ref{fig:mms1_tl_phi}, \fig{fig:mms1_lobatto_phi}, and \fig{fig:mms1_gauss_phi} plot the order of convergence of $E_{\phi}$ for the TL, SL Lobatto, and SL Gauss schemes, respectively, as a function of trial space degree and cell size.
Analogous data for $E_{T}$ is given in \figs{fig:mms1_tl_temp}{fig:mms1_gauss_temp}.
%
\begin{figure}[!htp]
\centering
\includegraphics[width=10cm,trim=0.25in  0.5in 0.75in 0.75in,clip=true]{chapter6_grey_radtran/Dissertation_Data/MMS2_TL_phi_L2.pdf}
\caption{Convergence of $E_{\phi}$ for the TL scheme in problem MMS1.}
\label{fig:mms1_tl_phi}
\end{figure}
%
%
\begin{figure}[!hbp]
\centering
\includegraphics[width=10cm,trim=0.25in  0.5in 0.75in 0.75in,clip=true]{chapter6_grey_radtran/Dissertation_Data/MMS2_SLXS_Lobatto_phi_L2.pdf}
\caption{Convergence of $E_{\phi}$ for the SL Lobatto scheme in problem MMS1.}
\label{fig:mms1_lobatto_phi}
\end{figure}
%
%
\begin{figure}[!htp]
\centering
\includegraphics[width=10cm,trim=0.25in  0.5in 0.75in 0.75in,clip=true]{chapter6_grey_radtran/Dissertation_Data/MMS2_SLXS_Gauss_phi_L2.pdf}
\caption{Convergence of $E_{\phi}$ for the SL Gauss scheme in problem MMS1.}
\label{fig:mms1_gauss_phi}
\end{figure}
%
\begin{figure}[!hbp]
\centering
\includegraphics[width=10cm,trim=0.25in  0.5in 0.75in 0.75in,clip=true]{chapter6_grey_radtran/Dissertation_Data/MMS2_TL_temp_L2.pdf}
\caption{Convergence of $E_{T}$ for the TL scheme in problem MMS1.}
\label{fig:mms1_tl_temp}
\end{figure}
%
%
\begin{figure}[!htp]
\centering
\includegraphics[width=10cm,trim=0.25in  0.5in 0.75in 0.75in,clip=true]{chapter6_grey_radtran/Dissertation_Data/MMS2_SLXS_Lobatto_temp_L2.pdf}
\caption{Convergence of $E_{T}$ for the SL Lobatto scheme in problem MMS1.}
\label{fig:mms1_lobatto_temp}
\end{figure}
%
%
\begin{figure}[!hbp]
\centering
\includegraphics[width=10cm,trim=0.25in  0.5in 0.75in 0.75in,clip=true]{chapter6_grey_radtran/Dissertation_Data/MMS2_SLXS_Gauss_temp_L2.pdf}
\caption{Convergence of $E_{T}$ for the SL Gauss scheme in problem MMS1.}
\label{fig:mms1_gauss_temp}
\end{figure}

We make several observations regarding \figs{fig:mms1_tl_phi}{fig:mms1_gauss_temp}.  
First, the TL scheme does not necessarily increase in accuracy with an increase in trial space degree.
TL convergence of $E_{\phi}$ is limited to second order in space for odd degree trial space DFEM and third order spatial convergence for even degree trial space schemes, behavior identical to that demonstrated in our neutron transport testing.
A similar limit exists for TL convergence of $E_T$, at most first order for odd trial space degree and second order for even trial space degree.
Second, SL Lobatto converges the $L^2$ norm of the radiation energy density $\propto P+1$, the same order of convergence achieved in our neutron transport test problems for the convergence of $L^2$ error of the angular and scalar fluxes.
However, SL Lobatto only converges $E_T \propto P$, not $P+1$, as was the case for the neutron transport interaction rate.
Finally, SL Gauss converges $E_{\phi}$ and $E_T$ $\propto P+1$, the same order of convergence as seen with neutron transport.

To understand the behaviors of SL Lobatto and SL Gauss completely, we now consider the $L^2$ like norm of cell average radiation energy density error, $E_{\phi_A}$ and cell average temperature error, $E_{T_A}$.
$E_{\phi_A}$ is approximated as:
\benum
E_{\phi_A} = \sqrt{
\sum_{c=1}^{N_{cell}}{ 
\frac{\Delta x}{2} 
\left( 
\frac{1}{2}\sum_{q=1}^{N_{qf}}{ w_q \widetilde{\phi}(s_q , t_{end})}  - \frac{1}{2}\sum_{q=1}^{N_{qf}}{ w_q \phi(s_q , t_{end})} 
\right)^2 
} 
} \pec
\eenum
with $N_{qf} = 2P + 7$, using Gauss quadrature.  $E_{T_A}$ is estimated in a similar fashion.
Convergence of $E_{\phi_A}$ is given in \fig{fig:mms1_lobatto_phi_A} for SL Lobatto and \fig{fig:mms1_gauss_phi_A} for SL Gauss.
Convergence of $E_{T_A}$ is given in \figs{fig:mms1_lobatto_temp_A}{fig:mms1_gauss_temp_A} for SL Lobatto and SL Gauss, respectively.
No definitive pattern emerges in the convergence of $E_{\phi_A}$ as a function of $P$ for SL Lobatto in \fig{fig:mms1_lobatto_phi_A} when considering linear through quartic polynomial trial spaces, though SL Lobatto convergence of $E_{\phi_A}$ looks to be $\propto 2P$ or slightly less.
SL Lobatto's apparent order of convergence for $E_{\phi_A}$ for TRT problems as shown in \fig{fig:mms1_lobatto_phi_A} is less than SL Lobatto convergence of $E_{\psi_A}$ for neutron transport (\figs{fig:multi_L2A_p1}{fig:multi_L2A_p4}), but not significantly. 
Similarly, the radiative transfer variant of SL Gauss does not converge $E_{\phi_A}$ for TRT simulations consistently $\propto 2P+1$ as SL Gauss converged $E_{\psi_A}$ neutron transport problems,
but it is close.  
Interestingly, the observed decreases in the convergence of $E_{\phi_A}$ for SL Lobatto and SL Gauss when applied to TRT relative to neutron transport convergence of $E_{\psi_A}$ does not hold when considering the convergence of $E_{T_A}$.  
SL Lobatto converges $E_{T_A}$ for TRT with the same order as SL Lobatto converges $E_{\psi_A}$ and $E_{IR_A}$ for neutron transport, $\propto 2P$.
SL Gauss actually exhibits an $E_{T_A}$ order of convergence that appears to exceed its neutron transport analog, converging $E_{T_A} \propto 2P+2$. \

\begin{figure}[!htp]
\centering
\includegraphics[width=9cm,trim=0.25in  0.5in 0.75in 0.75in,clip=true]{chapter6_grey_radtran/Dissertation_Data/MMS2_SLXS_Lobatto_phi_A.pdf}
\caption{Convergence of $E_{\phi_A}$ for the SL Lobatto scheme in problem MMS1.}
\label{fig:mms1_lobatto_phi_A}
\end{figure}
%
%
\begin{figure}[!hbp]
\centering
\includegraphics[width=9cm,trim=0.25in  0.1in 0.75in 0.5in,clip=true]{chapter6_grey_radtran/Dissertation_Data/MMS2_SLXS_Gauss_phi_A.pdf}
\caption{Convergence of $E_{\phi_A}$ for the SL Gauss scheme in problem MMS1.}
\label{fig:mms1_gauss_phi_A}
\end{figure}
%
%
\begin{figure}[!htp]
\centering
\includegraphics[width=10cm,trim=0.25in  0.5in 0.75in 0.75in,clip=true]{chapter6_grey_radtran/Dissertation_Data/MMS2_SLXS_Lobatto_temp_A.pdf}
\caption{Convergence of $E_{T_A}$ for the SL Lobatto scheme in problem MMS1.}
\label{fig:mms1_lobatto_temp_A}
\end{figure}
%
%
\begin{figure}[!hbp]
\centering
\includegraphics[width=10cm,trim=0.25in  0.5in 0.75in 0.75in,clip=true]{chapter6_grey_radtran/Dissertation_Data/MMS2_SLXS_Gauss_temp_A.pdf}
\caption{Convergence of $E_{T_A}$ for the SL Gauss scheme in problem MMS1.}
\label{fig:mms1_gauss_temp_A}
\end{figure}

It is our hypothesis that the apparent super convergence demonstrated by SL Gauss in converging $E_T$ is related to how we expand the Planckian in the same trial space as the DFEM temperature and radiation solutions. 
That is to say that we suspect the $N_P$ quadrature points of a given trial space degree SL Gauss scheme are significantly more accurate at integrating 
\benum
\B{i}(s) B(\widetilde{T}) \pec
\label{eq:planck_int}
\eenum
a degree $5P$ polynomial, than an $N_P$ Lobatto quadrature.
Though the $N_P$ point Gauss quadrature only exactly integrates polynomials of degree $2(P+1) -1$, it may be that the remaining terms are coincidentally very small in magnitude for a degree $5P$ polynomial of the particular nature of \eqt{eq:planck_int}.

%This plateauing of errors seen in the above plots is a result of two things.  
%First, the nested nature of our TRT solution algorithm.
%Due to the nested iterations, we must use the tightest tolerance on the innermost solve, the intensity update.
%We terminate the innermost solve after reaching a certain point-wise change tolerance, $\epsilon_{inner}$, of the zero-th angular moment of the angle integrated intensity: 
%\benum
%\max_{j} { \abs{ \frac{\phi_{j}^{(\ell+1)} - \phi_j^{(\ell)} }{\phi_{j}^{(\ell+1)} } }  } < \epsilon_{inner} \pep
%\eenum
%Similarly, the thermal iteration of SDIRK stage $s$ is terminated when every temperature solution at the DFEM interpolation points, $T_j$,  changes less than $\epsilon_{thermal}$,
%\benum
 %\max_{j} { { \abs{ \frac{ T_{j,s}^{(\ell+1)} - T_j^{(\ell)} }{T_{j}^{(\ell+1)} } }  } }< \epsilon_{thermal} \pep
%\eenum
%For this and all other MMS problems, $\epsilon_{inner} = 10^{-12}$ and $\epsilon_{thermal} = 10^{-10}$.
%Secondly, as the TRT equations are functions of space and time, it possible that the error of a given spatial discretization is smaller than the temporal error of the problem, thus a plateauing of the solution error as a function of spatial mesh refinement occurs.
%Thus, truly accurate solutions require the refinement of both spatial mesh and time step size.

\subsubsection{Trigonometric in  Space - Linear in Time - with Temperature Dependent Material Properties}

We now consider a problem with temperature dependent material properties, MMS2.  We impose the following solution:
\beanum
M(\mu_d) &=& \frac{1}{4\pi} \\
W_I(x) &=& 9 \cos\left( \frac{\pi x}{10} - \frac{\pi}{2} \right) + 3 \pec \\
W_T(x) &=&  5 \cos\left( \frac{\pi x}{10} - \frac{\pi}{2} \right) + 5 \pec \\
F(t) &=&  1 + .02t \pec
\eeanum
and define the following material properties:
\beanum
C_v &=& 0.2 + 0.01 T^3 \\
\sigma_a &=& \frac{10^4}{T^3} \\
\sigma_s &=& 0.5 \pep
\eeanum
In our problem, $x\in[0,10]$, $t\in[0,2]$, we use the three stage, third order accurate SDRIK method of Alexander \cite{alexander} with $\Delta t = 0.001$.  
In total , we consider four different DFEM schemes for this problem, SL Lobatto, SL Gauss, SLXS Lobatto, and SLXS Gauss.  
The methods denoted SL assume cell-wise constant opacities and heat capacities,  equal to the cell-wise volumetric average of that quantity.  
The SLXS schemes explicitly account for the within cell variation of temperature dependent material properties by evaluating $\mathbf{R}$ as in \eqt{eq:chap3_sl_react}.
$E_{\phi}$ for the SL Lobatto and SL Gauss schemes is plotted in \fig{fig:mms3_constant_lobatto_phi} and \fig{fig:mms3_constant_gauss_phi}, respectively.
\begin{figure}[!hbp]
\centering
\includegraphics[width=10cm,trim=0.25in  0.25in 0.75in 0.5in,clip=true]{chapter6_grey_radtran/Dissertation_Data/MMS3_Constant_XS_SL_Lobatto_phi_L2.pdf}
\caption{Convergence of $E_{\phi}$ for the SL Lobatto scheme in problem MMS2.}
\label{fig:mms3_constant_lobatto_phi}
\end{figure}
%
%
\begin{figure}[!htp]
\centering
\includegraphics[width=10cm,trim=0.25in  0.25in 0.75in 0.5in,clip=true]{chapter6_grey_radtran/Dissertation_Data/MMS3_Constant_XS_SL_Gauss_phi_L2.pdf}
\caption{Convergence of $E_{\phi}$ for the SL Gauss scheme in problem MMS2.}
\label{fig:mms3_constant_gauss_phi}
\end{figure}
\begin{figure}[!hbp]
\centering
\includegraphics[width=10cm,trim=0.25in  0.2in 0.75in 0.5in,clip=true]{chapter6_grey_radtran/Dissertation_Data/MMS3_Constant_XS_SL_Lobatto_temp_L2.pdf}
\caption{Convergence of $E_{T}$ for the SL Lobatto scheme in problem MMS2.}
\label{fig:mms3_constant_lobatto_temp}
\end{figure}
%
%
\begin{figure}[!htp]
\centering
\includegraphics[width=10cm,trim=0.25in  0.2in 0.75in 0.5in,clip=true]{chapter6_grey_radtran/Dissertation_Data/MMS3_Constant_XS_SL_Gauss_temp_L2.pdf}
\caption{Convergence of $E_{T}$ for the SL Gauss scheme in problem MMS2.}
\label{fig:mms3_constant_gauss_temp}
\end{figure}

Regardless of DFEM interpolation point type or trial space degree, assuming cell-wise constant material properties limits spatial convergence of $E_{\phi}$ to at most second order.
Confirming our suspicion that assuming cell-wise constant material properties limit the convergence of $E_T$ as well, we plot the convergence of $E_T$ for the SL Lobatto scheme in \fig{fig:mms3_constant_lobatto_temp} and for the SL Gauss scheme in \fig{fig:mms3_constant_gauss_temp}.
Figures \ref{fig:mms3_constant_lobatto_temp}-\ref{fig:mms3_constant_gauss_temp} verify the hypothesis we developed while discussing our neutron transport results: assuming a cell-wise constant opacity limits $L^2$ convergence of temperature to at most first order in space, regardless of DFEM trial space degree.

To verify that the high order of convergence observed for $E_{IR_A}$ in Section 3 does not translate to a high rate of spatial convergence for $E_{T_A}$, we consider \fig{fig:mms3_constant_lobatto_temp_A} and \fig{fig:mms3_constant_gauss_temp_A}.
Figure \ref{fig:mms3_constant_lobatto_temp_A} verifies that regardless of trial space degree, the SL Lobatto scheme assuming a cell-wise constant opacity for a problem with temperature or spatially varying opacities converges $E_{T_A}$ second order in space.
Likewise, \fig{fig:mms3_constant_gauss_temp_A} demonstrates the same is true for the SL Gauss scheme.
\begin{figure}[!hbp]
\centering
\includegraphics[width=9.5cm,trim=0.25in  0.2in 0.75in 0.5in,clip=true]{chapter6_grey_radtran/Dissertation_Data/MMS3_Constant_XS_SL_Lobatto_temp_A.pdf}
\caption{Convergence of $E_{T_A}$ for the SL Lobatto scheme in problem MMS2.}
\label{fig:mms3_constant_lobatto_temp_A}
\end{figure}
%
%
\begin{figure}[!htp]
\centering
\includegraphics[width=9.5cm,trim=0.25in  0.2in 0.75in 0.5in,clip=true]{chapter6_grey_radtran/Dissertation_Data/MMS3_Constant_XS_SL_Gauss_temp_A.pdf}
\caption{Convergence of $E_{T_A}$ for the SL Gauss scheme in problem MMS2.}
\label{fig:mms3_constant_gauss_temp_A}
\end{figure}

We now consider the convergence of SLXS Lobatto and SLXS Gauss.  First, we see that in \fig{fig:mms3_slxs_lobatto_e_phi}, SLXS Lobatto converges $E_{\phi} \propto P+1$, and in \fig{fig:mms3_slxs_gauss_e_phi}, SLXS Gauss converges $E_{\phi}$ $\propto P+1$ as well.
Moving on to the convergence of $E_T$, in \fig{fig:mms3_slxs_lobatto_e_t} SLXS Lobatto converges $E_T \propto P$, the same rate SL Lobatto converged $E_T$ for the TRT problem with constant material properties.  
In \fig{fig:mms3_slxs_gauss_e_t} SLXS Gauss converges $E_T$ very rapidly, and only the asymptotic convergence rate of linear and quadratic DFEM can be estimated, though both suggest SLXS Gauss converges $E_T \propto P+2$.
\begin{figure}[!hbp]
\centering
\includegraphics[width=10cm,trim=0.25in  0.25in 0.75in 0.75in,clip=true]{chapter6_grey_radtran/Dissertation_Data/MMS3_SLXS_Lobatto_phi_L2.pdf}
\caption{Convergence of $E_{\phi}$ for the SLXS Lobatto scheme in problem MMS2.}
\label{fig:mms3_slxs_lobatto_e_phi}
\end{figure}
%
%
\begin{figure}[!htp]
\centering
\includegraphics[width=10cm,trim=0.25in  0.25in 0.75in 0.75in,clip=true]{chapter6_grey_radtran/Dissertation_Data/MMS3_SLXS_Gauss_phi_L2.pdf}
\caption{Convergence of $E_{\phi}$ for the SLXS Gauss scheme in problem MMS2.}
\label{fig:mms3_slxs_gauss_e_phi}
\end{figure}
\begin{figure}[!hbp]
\centering
\includegraphics[width=10cm,trim=0.25in  0.25in 0.75in 0.75in,clip=true]{chapter6_grey_radtran/Dissertation_Data/MMS3_SLXS_Lobatto_temp_L2.pdf}
\caption{Convergence of $E_{T}$ for the SLXS Lobatto scheme in problem MMS2.}
\label{fig:mms3_slxs_lobatto_e_t}
\end{figure}
%
%
\begin{figure}[!htp]
\centering
\includegraphics[width=9.5cm,trim=0.25in  0.25in 0.75in 0.75in,clip=true]{chapter6_grey_radtran/Dissertation_Data/MMS3_SLXS_Gauss_temp_L2.pdf}
\caption{Convergence of $E_{T}$ for the SLXS Gauss scheme in problem MMS2.}
\label{fig:mms3_slxs_gauss_e_t}
\end{figure}
%
%
%
\begin{figure}[!hbp]
\centering
\includegraphics[width=9.5cm,trim=0.25in  0.4in 0.75in 0.75in,clip=true]{chapter6_grey_radtran/Dissertation_Data/MMS3_SLXS_SLXS_Lobatto_phi_A.pdf}
\caption{Convergence of $E_{\phi_A}$ for the SLXS Lobatto scheme in problem MMS2.}
\label{fig:mms3_slxs_lobatto_phi_a}
\end{figure}
%
%
\begin{figure}[!htp]
\centering
\includegraphics[width=10cm,trim=0.25in  0.25in 0.75in 0.75in,clip=true]{chapter6_grey_radtran/Dissertation_Data/MMS3_SLXS_Gauss_phi_A.pdf}
\caption{Convergence of $E_{\phi_A}$ for the SLXS Gauss scheme in problem MMS2.}
\label{fig:mms3_slxs_gauss_phi_a}
\end{figure}
\begin{figure}[!hbp]
\centering
\includegraphics[width=10cm,trim=0.25in  0.25in 0.75in 0.75in,clip=true]{chapter6_grey_radtran/Dissertation_Data/MMS3_SLXS_Lobatto_temp_A.pdf}
\caption{Convergence of $E_{T_A}$ for the SLXS Lobatto scheme in problem MMS2.}
\label{fig:mms3_slxs_lobatto_t_a}
\end{figure}
%
%
\begin{figure}[!htp]
\centering
\includegraphics[width=10cm,trim=0.25in  0.25in 0.75in 0.75in,clip=true]{chapter6_grey_radtran/Dissertation_Data/MMS3_SLXS_Gauss_temp_A.pdf}
\caption{Convergence of $E_{T_A}$ for the SLXS Gauss scheme in problem MMS2.}
\label{fig:mms3_slxs_gauss_t_a}
\end{figure}

Focusing now on the convergence of cell average error quantities, we first consider $E_{\phi_A}$ for SLXS Lobatto and SLXS Gauss in \fig{fig:mms3_slxs_lobatto_phi_a} and \fig{fig:mms3_slxs_gauss_phi_a}, respectively.
Both SLXS Lobatto and SLXS Gauss converge $E_{\phi_A}$ at an order less than or at most equal to the order their respective neutron transport analogs converged $E_{\psi_A}$.
However, since both methods require only a small amount of mesh refinement to reach an error level approximately equal to our temperature tolerance, it is difficult to establish with certainty the order of convergence of either method for higher $P$.
For completeness, we include convergence plots of $E_{T_A}$ for SLXS Lobatto and SLXS Gauss respectively in \fig{fig:mms3_slxs_lobatto_t_a} and \fig{fig:mms3_slxs_gauss_t_a}.
All we definitively conclude from \figs{fig:mms3_slxs_lobatto_t_a}{fig:mms3_slxs_gauss_t_a} is that both SLXS Lobatto and SLXS Gauss experience increases in convergence of $E_{T_A}$ with increases in trial space degree.

In \figs{fig:mms1_tl_phi}{fig:mms3_slxs_gauss_t_a}, the plateauing of errors that appears in some plots is a result of our relative convergence tolerances for both temperature and angle integrated intensity. 
Given our convergence criteria, we might expect a plateau, $E_{T,min}$ of approximately
\benum
E_{T,min} \int_{x_{1/2} }^{x_{N_{cell}+1/2}}{ \epsilon_T \left[ F_t(t_{end}) W_T(x) \right] dx} \pep
\label{eq:err_plateau}
\eenum
Applying \eqt{eq:err_plateau} to MMS1, $E_{T,min} = 4.6\times 10^{-8}$, which is consistent with the location of the $E_T$ error plateau.
In \eqt{eq:err_plateau}, we are implicitly assuming that the error in $\phi$ is significantly less than the error in temperature. 
Alternatively, we could consider resulted generated with less stringent $\epsilon_T$ and $\epsilon_{\phi}$.
In \fig{fig:low_tol_slxs_gauss_temp} and \fig{fig:low_tol_slxs_lobatto_phi}, we give the results for $E_{T}$ convergence for the SLXS Gauss scheme and $E_{\phi}$ convergence for the SLXS Lobatto scheme for MMS2, but use $\epsilon_T=10^{-8}$ and $\epsilon_{\phi} = 10^{-10}$.  
Comparing the $E_T$ plateau in \fig{fig:low_tol_slxs_gauss_temp} of $\approx 5 \times 10^{-7}$ we see a factor of roughly 1000 increase compared to the plateau observed in \fig{fig:mms3_slxs_gauss_e_t}
 for the same quantity, equivalent to the relative relaxations of $\epsilon_T$ and $\epsilon_{\phi}$.
Likewise, the $E_{\phi}$ plateau of $\approx 10^{-5}$ in \fig{fig:low_tol_slxs_lobatto_phi} is roughly 1000 times greater than the same plateau observed in \fig{fig:mms3_slxs_lobatto_e_phi}.

\begin{figure}[!htp]
\centering
\includegraphics[width=9cm,trim=0.5in  0.5in 1in 0.75in,clip=true]{chapter6_grey_radtran/Dissertation_Data/MMS3_Low_Tol_SLXS_Gauss_temp_L2.pdf}
\caption{Convergence of $E_{T}$ for the SLXS Gauss scheme in problem MMS2 using $\epsilon_T = 10^{-8}$ and $\epsilon_{\phi}=10^{-10}$.}
\label{fig:low_tol_slxs_gauss_temp}
\end{figure}
%
%
\begin{figure}[!hbp]
\centering
\includegraphics[width=9cm,trim=0.5in  0.5in 1in 0.75in,clip=true]{chapter6_grey_radtran/Dissertation_Data/MMS3_Low_Tol_SLXS_Lobatto_phi_L2.pdf}
\caption{Convergence of $E_{\phi}$ for the SLXS Lobatto scheme in problem MMS2 using $\epsilon_T = 10^{-8}$ and $\epsilon_{\phi}=10^{-10}$.}
\label{fig:low_tol_slxs_lobatto_phi}
\end{figure}

\subsubsection{Constant in Time - Trigonometric in Space - with Temperature Dependent Material Properties}

We now consider a problem that is constant in time and varies as a cosine in space with temperature dependent material properties.  
Though very similar to MMS2, we hope that by considering a truly steady state problem, we can study the spatial error of our DFEM schemes without temporal error interference.
We impose a solution of the form:
\beanum
M(\mu_d) &=& \frac{1}{4\pi} \\
W_I(x) &=& 19 \cos\left( \frac{\pi x}{2} \right) + 20 \pec \\
W_T(x) &=&  15 \cos\left( \frac{\pi x}{2}  \right) + 20 \pec \\
F(t) &=&  10
\eeanum
and define the following material properties:
\beanum
C_v &=& 0.1 + 0.2 T^2 \\
\sigma_a &=& \frac{5}{T^2} \\
\sigma_s &=& 0.01 \pep
\eeanum
Repeating as we have before, we first consider the convergence of $E_{\phi}$ for SLXS Lobatto and SLXS Gauss in \fig{fig:constant_time_lobatto_phi} and \fig{fig:constant_time_gauss_phi}.
\begin{figure}[!htp]
\centering
\includegraphics[width=10cm,trim=0.25in  0.2in 0.75in 0.5in,clip=true]{chapter6_grey_radtran/Dissertation_Data/Constant_Time_SLXS_Lobatto_phi_L2.pdf}
\caption{Convergence of $E_{\phi}$ for SLXS Lobatto scheme for steady state test problem.}
\label{fig:constant_time_lobatto_phi}
\end{figure}
%
%
\begin{figure}[!hbp]
\centering
\includegraphics[width=10cm,trim=0.25in  0.2in 0.75in 0.5in,clip=true]{chapter6_grey_radtran/Dissertation_Data/Constant_Time_SLXS_Gauss_phi_L2.pdf}
\caption{Convergence of $E_{\phi}$ for SLXS Gauss scheme, for steady state test problem.}
\label{fig:constant_time_gauss_phi}
\end{figure}
As before, SLXS Lobatto converges $E_{\phi} \propto P+1$.  
However, SLXS Gauss appears to converges $E_{\phi} \propto P+2$.
It is not clear why SLXS Gauss convergence of $E_{\phi}$ increases when moving from the time dependent MMS1 and MMS2 problems to the steady-state test problem.

Now considering the convergence of $E_{T}$, the steady-state problem confirms SLXS Lobatto converges $E_T$ $\propto P$, shown in \fig{fig:constant_time_lobatto_t}, and SLXS Gauss converges $E_T \propto P+2$, as shown in \fig{fig:constant_time_gauss_t}.
\begin{figure}[!htp]
\centering
\includegraphics[width=10cm,trim=0.25in  0.2in 0.75in 0.5in,clip=true]{chapter6_grey_radtran/Dissertation_Data/Constant_Time_SLXS_Lobatto_temp_L2.pdf}
\caption{Convergence of $E_{T}$ for SLXS Lobatto scheme, for steady state test problem.}
\label{fig:constant_time_lobatto_t}
\end{figure}
%
\begin{figure}[!hbp]
\centering
\includegraphics[width=9.5cm,trim=0.25in  0.2in 0.75in 0.5in,clip=true]{chapter6_grey_radtran/Dissertation_Data/Constant_Time_SLXS_Gauss_temp_L2.pdf}
\caption{Convergence of $E_{T}$ for SLXS Gauss scheme, for steady state test problem.}
\label{fig:constant_time_gauss_t}
\end{figure}

We now examine the convergence of $E_{\phi_A}$.  
The estimated order of convergence of $E_{\phi_A}$ in \fig{fig:constant_time_lobatto_phi_a} for SLXS Lobatto, appears to be $\propto 2P$, greater than the results from MMS1 and MMS2, but equal to what we would have hypothesized from neutron transport.  
%Figure \ref{fig:constant_time_lobatto_phi_a} and \fig{fig:mms3_slxs_lobatto_phi_a} giving different estimates of $E_{\phi_A}$ convergence is most likely due to a lack of data points available prior to reaching our iterative tolerances, preventing a clear inference of asymptotic convergence rate.
Similarly the convergence of $E_{\phi_A}$ for SLXS Gauss estimated in \fig{fig:constant_time_gauss_phi_a} appears to be greater, $\propto 2P+2$, than the SLXS Gauss order of convergence for $E_{\phi_A}$, $<2P+1$, given in \fig{fig:mms3_slxs_gauss_phi_a}.
\begin{figure}[!hbp]
\centering
\includegraphics[width=10cm,trim=0.25in  0.2in 0.75in 0.5in,clip=true]{chapter6_grey_radtran/Dissertation_Data/Constant_Time_SLXS_Lobatto_phi_A.pdf}
\caption{Convergence of $E_{\phi_A}$ for SLXS Lobatto scheme, for steady state test problem.}
\label{fig:constant_time_lobatto_phi_a}
\end{figure}
%
%
\begin{figure}[!htp]
\centering
\includegraphics[width=10cm,trim=0.25in  0.2in 0.75in 0.5in,clip=true]{chapter6_grey_radtran/Dissertation_Data/Constant_Time_SLXS_Gauss_phi_A.pdf}
\caption{Convergence of $E_{\phi_A}$ for SLXS Gauss scheme, for steady state test problem.}
\label{fig:constant_time_gauss_phi_a}
\end{figure}

Finally, we consider the convergence of $E_{T_A}$.  
In \fig{fig:constant_time_lobatto_t_a} SLXS Lobatto converges $E_{T_A}$ $\propto 2P$, and in \fig{fig:constant_time_gauss_t_a}
SLXS Gauss converges $E_{T_A} \propto 2P+2$.  
Both of these convergence rates are higher than those observed in any of our previous MMS test problems.
The SLXS Lobatto $E_{T_A}$ convergence rate is equal to the SLXS Lobatto convergence rates for $E_{\psi_A}$ and $E_{IR_A}$, whereas the SLXS Gauss $E_{T_A}$ convergence rate is greater than any convergence rate observed for SLXS Gauss for neutron transport.
\begin{figure}[!htp]
\centering
\includegraphics[width=9.5cm,trim=0.25in  0.25in 0.75in 0.5in,clip=true]{chapter6_grey_radtran/Dissertation_Data/Constant_Time_SLXS_Lobatto_temp_A.pdf}
\caption{Convergence of $E_{T_A}$ for SLXS Lobatto scheme, for steady state test problem.}
\label{fig:constant_time_lobatto_t_a}
\end{figure}
%
%
\begin{figure}[!hbp]
\centering
\includegraphics[width=9.5cm,trim=0.25in  0.25in 0.75in 0.5in,clip=true]{chapter6_grey_radtran/Dissertation_Data/Constant_Time_SLXS_Gauss_temp_A.pdf}
\caption{Convergence of $E_{T_A}$ for SLXS Gauss scheme, for steady state test problem.}
\label{fig:constant_time_gauss_t_a}
\end{figure}

%\pagebreak
\newpage
\subsubsection{Constant in Space - Trigonometric in Time}
\label{sec:time_convergence}


We now verify the asymptotic order of convergence of each SDIRK scheme: implicit Euler (IE); the two stage, second order accurate scheme of Alexander (2-2); and the three stage, third order accurate scheme of Alexander (3-3).
To simplify the process, we consider a problem with constant spatial dependence:
\beanum
M(\mu_d) &=& \frac{1}{4\pi} \\
W_I(x) &=& \frac{10}{4\pi} \\
W_T(x) &=&  10 \\
F(t) &=& 45 \cos\left( \pi t \right) + 46 \pec
\eeanum
$t \in[0,1]$, $\sigma_s = 0.1$, $\sigma_a = 2.5$, $C_v = 0.2$, $x\in[0,10]$ discretized with 10 equally spaced cells.  Convergence of $E_{\phi}$ as a function of $\Delta t$ for the IE, 2-2, and 3-3 time differencing schemes is given in \fig{fig:e_phi_time}.
\begin{figure}[!htp]
\centering
\includegraphics[width=10cm,trim=0.25in  0.2in 0.75in 0.5in,clip=true]{chapter6_grey_radtran/Dissertation_Data/Time_Integrators_Convergence_Phi.pdf}
\caption{Convergence of $E_{\phi}$ for different time integrators as a function of $\Delta t$.}
\label{fig:e_phi_time}
\end{figure}
As expected, the IE converges 1st order in time, the 2-2 scheme converges second order, and the Alexander 3-3 scheme converges third order in time.
The same data for temperature is given in \fig{fig:big_dt}.  
Though the 2-2 and 3-3 schemes are more accurate than IE, it took a very long time for the 2-2 and 3-3 schemes to demonstrate their respective asymptotic orders of convergence for $E_T$ as compared to $E_{\phi}$.
\begin{figure}[!htp]
\centering
\includegraphics[width=10cm,trim=0.25in  0.2in 0.75in 0.5in,clip=true]{chapter6_grey_radtran/Dissertation_Data/Time_Integrators_Convergence_Temperature.pdf}
\caption{Convergence of $E_{T}$ for different time integrators as a function of $\Delta t$.}
\label{fig:big_dt}
\end{figure}

\section{Thermal Radiative Transfer Simulations without Analytic Solutions}
\label{sec:marshak_waves}

\subsection{``Unity'' Marshak Wave Problem}

We now consider our most realistic test problem, a Marshak wave test problem.
Originally presented by Ober and Shadid for radiative diffusion \cite{ober_shadid} we compare our $S_2$ solution that uses Gauss quadrature in angle to the solutions presented in \cite{ober_shadid}.
With an $S_2$ Gauss quadrature in angle, the discrete ordinates method is equivalent to a $P_1$ angular discretization in slab geometry \cite{s2sa}.
The radiative diffusion solution of \cite{ober_shadid} could also be considered a $P_0$ approximation.
Thus while not exact, our $P_1$ equivalent solution is nearly equal to the $P_0$ solutions presented in \cite{ober_shadid}.
Like most thermal radiative transfer problems, no analytic solution exists.  
As such, we focus on qualitative comparisons of how DFEM trial space degree, mesh refinement, and time step refinement affect the radiation energy density and material temperature solutions.

The Marshak wave problem consists of an initially cold slab that is heated from the left by an incident radiation intensity, with a vacuum boundary condition on the right.
Again, the problem is dimensionless, $a=c=C_v=1$; $x\in[0,1]$; and we advance the solution from $t=0$ to $t=1$.
Initially, the slab is in thermal equilibrium, and $T=\left( 10^{-5} \right)^{1/4}$.
There is no scattering, $\sigma_s = 0$, and the absorption opacity is temperature dependent, $\sigma_a = \frac{1}{T^3}$.

We use an initial mesh of 20 cells, but also consider results using meshes of 80, 320, and 1280 cells.
Unless otherwise noted, for all simulations we use the 2-2 Alexander scheme.
We begin with a minimum time step of $\Delta t = 5\times 10^{-4}$ and increase the time step size by 10\% until we reach a maximum time step size of $\Delta t = 10^{-2}$.
For our time refinement studies, we divide the minimum and maximum time step sizes both by a factor of 4, 16, or 64.
We consider linear, quadratic, cubic and quartic SLXS Lobatto and SLXS Gauss DFEM schemes.
To demonstrate that assuming a cell-wise constant opacity results in a bladed radiative transfer temperature solution, we also consider linear SL Lobatto with a cell-wise volumetric average opacity.

We first investigate the effect, if any, of assuming a cell-wise constant opacity.  Figure \ref{fig:bladed_rad_profile} shows the linear SL Lobatto radiation energy density solution at $t=1.0$.
Except for the effects of a very coarse spatial mesh when using 20 cells, the radiation energy density solution is effectively smooth, and comparable to the results published in \cite{ober_shadid}.  
The full temperature solution is shown in \fig{fig:bladed_t_profile_full}.
Clearly, the large, non-monotonic discontinuities (blading) observed in the neutron transport interaction rate profile are present in the TRT temperature profile.
In \fig{fig:bladed_t_profile_full}, mesh refinement reduces the magnitude of the temperature solution blading but does not eliminate blading.
To emphasize that blading is not eliminated with mesh refinement, consider \fig{fig:bladed_t_profile_zoom}, that zooms in on the material temperature solution near the Marshak wavefront.
\begin{figure}[!htp]
\centering
\includegraphics[width=9.5cm,trim=1.2in  0.2in 0.75in 0.5in,clip=true]{chapter6_grey_radtran/Dissertation_Data/Reorder_Blading_Radiation_Full_MultiCell.pdf}
\caption{Linear SL Lobatto radiation solution for the Marshak wave problem assuming cell-wise constant volumetric averaged opacities.}
\label{fig:bladed_rad_profile}
\end{figure}
%
%
\begin{figure}[!hbp]
\centering
\includegraphics[width=9.5cm,trim=1.2in  0.2in 0.75in 0.5in,clip=true]{chapter6_grey_radtran/Dissertation_Data/Reorder_Blading_Temperature_Full_MultiCell.pdf}
\caption{Linear SL Lobatto temperature solution for the Marshak wave problem assuming cell-wise constant volumetric averaged opacities.}
\label{fig:bladed_t_profile_full}
\end{figure}
%
%
\begin{figure}[!htp]
\centering
\includegraphics[width=9.5cm,trim=1.2in  0.2in 0.75in 0.5in,clip=true]{chapter6_grey_radtran/Dissertation_Data/Reorder_Blading_Temperature_Zoom_MultiCell.pdf}
\caption{Linear SL Lobatto temperature solution for the Marshak wave problem near the wavefront.}
\label{fig:bladed_t_profile_zoom}
\end{figure}
Clearly, \figs{fig:bladed_rad_profile}{fig:bladed_t_profile_zoom}, in addition to the limited order of convergence results given \fig{fig:mms3_constant_lobatto_temp} and \fig{fig:mms3_constant_gauss_temp} demonstrate that the assumption of a cell-wise constant opacity is not appropriate for thermal radiative transfer simulations with temperature dependent material properties.

For comparison, consider the linear SLXS Lobatto radiation energy density solution in \fig{fig:linear_slxs_full_rad} and the temperature solution in \fig{fig:linear_slxs_full_temp} using 80 mesh cells.
Visually, there is little difference between the SL Lobatto and SLXS Lobatto angle integrated intensity solutions.
However, this is not the case when examining the material temperature solutions.  Figure \ref{fig:linear_slxs_full_temp} does not exhibit any blading.

%
\begin{figure}[!hbp]
\centering
\includegraphics[width=9.5cm,trim=1.2in  0.2in 0.75in 0.5in,clip=true]{chapter6_grey_radtran/Dissertation_Data/SLXS_Lobatto_80_Cells_Radiation.pdf}
\caption{Linear SLXS Lobatto angle integrated intensity solution with 80 cells.}
\label{fig:linear_slxs_full_rad}
\end{figure}
%
%
\begin{figure}[!htp]
\centering
\includegraphics[width=9.5cm,trim=1.2in  0.2in 0.75in 0.5in,clip=true]{chapter6_grey_radtran/Dissertation_Data/SLXS_Lobatto_80_Cells_Temperature.pdf}
\caption{Linear SLXS Lobatto temperature solution with 80 cells.}
\label{fig:linear_slxs_full_temp}
\end{figure}
%

We now consider the effects of spatial mesh refinement on TRT solutions.  
We first consider the linear SLXS Lobatto scheme, looking at a zoom in near the wavefront of the radiation profile in \fig{fig:lobatto_convergence_rad} and the material temperature profile in \fig{fig:lobatto_convergence_temp}
\begin{figure}[!hbp]
\centering
\includegraphics[width=10cm,trim=1.0in  0.2in 0.5in 0.5in,clip=true]{chapter6_grey_radtran/Dissertation_Data/Reorder_Marshak_Zoom_Radiation_SL_Lobatto_P1_Cell_Refinement.pdf}
\caption{Linear SLXS Lobatto radiation solution near wavefront with increasing spatial mesh refinement.}
\label{fig:lobatto_convergence_rad}
\end{figure}
%
%
\begin{figure}[!htp]
\centering
\includegraphics[width=10cm,trim=1.0in  0.2in 0.5in 0.5in,clip=true]{chapter6_grey_radtran/Dissertation_Data/Reorder_Marshak_Zoom_Temperature_SL_Lobatto_P1_Cell_Refinement.pdf}
\caption{Linear SLXS Lobatto temperature solution near wavefront with increasing spatial mesh refinement.}
\label{fig:lobatto_convergence_temp}
\end{figure}
Though the changes are subtle, we can see that mesh refinement actually changes the location of both the radiation and temperature profile discontinuity, with the changes more prominent in the material temperature profile, \fig{fig:lobatto_convergence_temp}.
The changes in the material temperature profile are more pronounced, even when moving from 320 to 1280 mesh cells, than in the radiation profile most likely due to SLXS Lobatto's first order convergence of $E_{T}$.
Plots similar to \fig{fig:lobatto_convergence_rad} and \fig{fig:lobatto_convergence_temp} are provided in \fig{fig:gauss_convergence_rad} and \fig{fig:gauss_convergence_temp}, respectively for the quartic SLXS Gauss scheme.
\begin{figure}[!hbp]
\centering
\includegraphics[width=10cm,trim=1.0in  0.2in 0.5in 0.5in,clip=true]{chapter6_grey_radtran/Dissertation_Data/Reorder_Marshak_Zoom_Radiation_SL_Gauss_P4_Cell_Refinement.pdf}
\caption{Quartic SLXS Gauss radiation solution near wavefront with increasing spatial mesh refinement.}
\label{fig:gauss_convergence_rad}
\end{figure}
%
%
\begin{figure}[!htp]
\centering
\includegraphics[width=10cm,trim=1.0in  0.2in 0.5in 0.5in,clip=true]{chapter6_grey_radtran/Dissertation_Data/Reorder_Marshak_Zoom_Temperature_SL_Gauss_P4_Cell_Refinement.pdf}
\caption{Quartic SLXS Gauss temperature solution near wavefront with increasing spatial mesh refinement.}
\label{fig:gauss_convergence_temp}
\end{figure}
Due to the SLXS Gauss' high order of spatial convergence, few changes are noticeable with mesh refinement, except when moving from 20 to 80 cells.
However, when moving from 20 to 80 cells, the Gibbs' phenomena near the solution discontinuity are no longer visible, except for a very small negativity in the temperature solution.
All solutions for this spatial mesh refinement study use the finest time step sizes.

We now examine the effect of increasing DFEM trial space degree, on a fixed mesh of 320 spatial cells using SLXS Lobatto, focusing on the region near the Marshak wavefront.
In \fig{fig:p_convergence_rad} we plot the angle integrated intensity profile and plot the temperature profile in \fig{fig:p_convergence_temp}
\begin{figure}[!htp]
\centering
\includegraphics[width=10cm,trim=1.0in  0.2in 0.5in 0.5in,clip=true]{chapter6_grey_radtran/Dissertation_Data/Pointless_Marshak_Zoom_Radiation_Lobatto_P_Refinement.pdf}
\caption{SLXS Lobatto radiation energy density solution with 320 cells for different $P$ near Marshak wavefront.}
\label{fig:p_convergence_rad}
\end{figure}
%
\begin{figure}[!hbp]
\centering
\includegraphics[width=10cm,trim=1.0in  0.2in 0.5in 0.5in,clip=true]{chapter6_grey_radtran/Dissertation_Data/Pointless_Marshak_Zoom_Temperature_Lobatto_P_Refinement.pdf}
\caption{SLXS Lobatto material temperature solution with 320 cells for different $P$ near Marshak wavefront.}
\label{fig:p_convergence_temp}
\end{figure}
Increasing $P$ make the wavefront in the radiation profile sharper, but at a resolution of 320 cells, none of the $P$ considered result in a visually continuous solution.
The most notable changes with increase $P$ come in the temperature profile, where linear SLXS Lobatto does not form a sharp interface, whereas all of the higher $P$ schemes capture the non-smooth transition more accurately.

Before looking at very high spatial resolution solutions, we first consider the effect of time step refinement in \fig{fig:time_refinement_rad} for the radiation profile and in \fig{fig:time_refinement_temp}.
Both \fig{fig:time_refinement_rad} and \fig{fig:time_refinement_temp} use a quartic SLXS Gauss spatial discretization with 1280 cells.
The effects of decreasing time step size are non-trivial near the wavefront.
As seen in \fig{fig:time_refinement_rad}, at lower time resolutions, $\Delta t$ and $\frac{\Delta t}{4}$, the wavefront is visibly not uniform concave down, with several ``wiggles'' in the radiation profile in the heated region of the slab.
Additionally, increased temporal resolution causes the discontinuity at the leading edge of the wavefront to sharpen.
In \fig{fig:time_refinement_temp}, the effects of increased time resolution are the same as in \fig{fig:time_refinement_rad}.
However, increased time resolution more noticeably sharpens the discontinuity in the temperature profile than it eliminates wavefront wiggles, though in \fig{fig:time_refinement_temp} the $\Delta t$ curve is not strictly concave down in the heated region of the slab near the wavefront.
%
%
\begin{figure}[!htp]
\centering
\includegraphics[width=10cm,trim=1.0in  0.2in 0.5in 0.5in,clip=true]{chapter6_grey_radtran/Dissertation_Data/Time_Refinement_Zoom_Radiation.pdf}
\caption{Quartic SLXS Lobatto radiation energy density solution with 1280 cells for different time refinements near the Marshak wavefront.}
\label{fig:time_refinement_rad}
\end{figure}
%
%
\begin{figure}[!hbp]
\centering
\includegraphics[width=10cm,trim=1.0in  0.2in 0.5in 0.5in,clip=true]{chapter6_grey_radtran/Dissertation_Data/Time_Refinement_Zoom_Temperature.pdf}
\caption{Quartic SLXS Lobatto temperature solution with 1280 cells for different time refinements near the Marshak wavefront.}
\label{fig:time_refinement_temp}
\end{figure}

\pagebreak

We now discuss highly resolved $S_2$ solutions to the Marshak wave problem.
Our hope is that with sufficient spatial resolution and higher order DFEM, we are able to resolve transport boundary layers.
Our highest resolution simulation uses ten thousand spatial mesh cells.
Given the initial, cold temperature of the slab is roughly $T=0.056$,  $\sigma_t = \sigma_a = \frac{1}{T^3}$, then the total slab optical thickness is roughly 5700 MFP thick, and when using ten though cells, each mesh cell is roughly $0.57$ MFP thick.  
As noted by Larsen, Morel, and Miller \cite{thick_diffusion_larsen}, this type of mesh spacing is neither optically thick nor thin.
To answer whether ten thousand mesh cells is sufficient, we first consider \fig{fig:res_zoom_comparison} where we compare the results of cubic SLXS Lobatto schemes that use
\begin{enumerate}
\item ten thousand spatial cells, with ten thousand time steps and the 3-3 SDIRK scheme,
\item ten thousand spatial cells, with one thousand time steps and the 3-3 SDIRK scheme, and 
\item 1280 spatial cells, with six thousand time steps of the 3-3 SDIRK scheme.
\end{enumerate} 
\begin{figure}[!htp]
\centering
\includegraphics[width=13cm,trim=1.0in  0.75in 1.0in 1.0in,clip=true]{chapter6_grey_radtran/Dissertation_Data/Zoom_10k_Phi.pdf}
\caption{Plot of the radiation energy density on a logarithmic scale for different high resolution simulations near the Marshak wavefront.}
\label{fig:res_zoom_comparison}
\end{figure}
Missing' segments in \fig{fig:res_zoom_comparison} are caused by negative radiation energy densities.
The twelve-hundred cell simulation has only 1 negative node.
The ten thousand cell simulation that uses one thousand time steps has a total of 8 negative nodes; one entire cell has a negative radiation energy density, and two other cells have at least one node with a negative radiation energy density.
Since the ten thousand cell simulation with ten thousand time steps does not have any negative radiation energy densities, it is clear then that temporal resolution is also required.
Unfortunately, noting that the radiation energy density jumps four orders of magnitude for $x\in[0.387 0.388]$, our highest resolution run only had a total of ten cells available to resolve the transport boundary layer.
In \fig{fig:boundary_layer}, we plot the angular intensities for $\mu=\pm\frac{1}{\sqrt{3}}$ of the ten thousand cell, ten thousand time step simulation.  
\begin{figure}[!htp]
\centering
\includegraphics[width=11cm,trim=0.5in  0.0in 0.5in 0.5in,clip=true]{chapter6_grey_radtran/Dissertation_Data/50_Cells_at_Wavefront_Intensity_Log.pdf}
\caption{Plot of the angular intensity on a logarithmic scale near the transport boundary layer at the Marshak wave front.}
\label{fig:boundary_layer}
\end{figure}
Clearly the radiation traveling from the hot to cold region, $\mu=\frac{1}{\sqrt{3}}$ has the largest boundary layer, but even with less than ideal spatial resolution, the rapid rise in angular intensity appears to be smooth, suggesting we have resolved the radiation boundary layer.

Finally, we consider higher $S_N$ solutions to the Marshak wave problem.  First, we compare the $S_8$ solution to the $S_2$ solution for material temperature in \fig{fig:s2_vs_s8_temperature} and for radiation energy density in \fig{fig:s2_vs_s8_radiation}.  The $S_8$ solution in \figs{fig:s2_vs_s8_temperature}{fig:s2_vs_s8_radiation} was generated using quartic SLXS Gauss with five thousand spatial cells, the Alexander 2-2 SDIRK scheme and approximately ten thousand time steps.
\begin{figure}[!htp]
\centering
\includegraphics[width=11cm,trim=1.0in  0.5in 0.2in 0.6in,clip=true]{chapter6_grey_radtran/S8_vs_S2_Material_Temperature.pdf}
\caption{$S_8$ and $S_2$ material temperature profiles.}
\label{fig:s2_vs_s8_temperature}
\end{figure}
%
\begin{figure}[!htp]
\centering
\includegraphics[width=11cm,trim=1.0in  0.3in 0.2in 0.5in,clip=true]{chapter6_grey_radtran/S8_vs_S2_Radiation_Energy_Density.pdf}
\caption{$S_8$ and $S_2$ radiation energy density profiles.}
\label{fig:s2_vs_s8_radiation}
\end{figure}
The $S_8$ solution exhibits many of the qualitative features we would expect a transport solution to exhibit versus the $S_2$ solution which is very close to the diffusion solution.
For example, we expect the transport solution to have a higher temperature solution near the problem boundary, with a more rapid drop in both the material temperature and radiation energy density/angle integrated intensity solutions relative to a diffusion solution.
Additionally, we expect the transport material temperature and radiation energy density solution to penetrate farther into the slab than the diffusion solution, but with a less steep gradient.
Figures \ref{fig:s2_vs_s8_radiation}-\ref{fig:s2_vs_s8_temperature} both exhibit this behavior, caused by the transport solution becoming more and more like $\delta(\mu-1)$ in optically thick regions. 
However, we did not expect the non-smooth features near $x=0.2$.
We suspect this is caused by the rapid attenuation of the most glancing $\mu$ from the incident boundary conditions.
To verify this, we plot the $S_8$ intensity solution in \fig{fig:s8_intensity_full}.
\begin{figure}[!htp]
\centering
\includegraphics[width=16cm,trim=1.5in  0.5in 0.2in 1in,clip=true]{chapter6_grey_radtran/S8_Intensity_SemiLogy.pdf}
\caption{Log plot of $S_8$ intensities for Marshak wave problem.}
\label{fig:s8_intensity_full}
\end{figure}

We now zoom in to the boundary layer intensities near $x=0.2$ and near the thermal wave front.
In \fig{fig:s8_zoom_glance}, it is clear that the intensity in the direction of $\mu=+0.1834$ experiences a rapid variation as the incident flux from the boundary is rapidly attenuated, and the isotropic emission from the heated regions of the slabs becomes the main contributor to $I(\mu_d = +0.1834)$.
It is also clear that despite having 25 cells with quartic DFEM in the region $x\in[.18,.185]$, the factor $\approx 7\times$ step drop in $I(\mu_d = +0.1834)$ cannot be fully resolved. 
\begin{figure}[!ht]
\centering
\includegraphics[width=12cm,trim=0.5in  0.2in 0.5in 0.5in,clip=true]{chapter6_grey_radtran/Dissertation_Data/S8_pos_mu_glance_boundary_layer_log.pdf}
\caption{Logarithmic plot of intensity near glancing $\mu=+0.1834$ boundary layer. }
\label{fig:s8_zoom_glance}
\end{figure}

Near the hot/cold material interface, all but the most glancing $\mu_d > 0$ experience a rapid variation (greater than a factor of 1000 reduction) in angular intensity.
The more glancing $\mu_d$, the sooner the transition, relative to the left boundary.
Given the smooth, profiles for $\mu_d > 0.2$, it also appears that we are able to fully resolve these boundary layers. 
\begin{figure}[!ht]
\centering
\includegraphics[width=16cm,trim=1.5in  0.2in 0.5in 0.75in,clip=true]{chapter6_grey_radtran/Dissertation_Data/S8_thermal_wavefront_boundary_layer.pdf}
\caption{Logarithmic plot of intensity boundary layers near thermal wavefront.}
\label{fig:s8_zoom_thermal_wavefront}
\end{figure}

Our final look at the Marshak wave problem investigates the structure of an $S_{32}$ solution with 1000 spatial cells, quartic SLXS Gauss, and five thousand time steps using the 2-2 time integration scheme.
The material temperature solution is plotted in \fig{fig:s8_vs_s32_temperature} against the $S_8$ solution that uses quartic SLXS Gauss,  five thousand spatial cells, and ten thousand 2-2 time integration steps.
Likewise, the angle integrated intensity solutions are compared in \fig{fig:s8_vs_s32_radiation}.
\begin{figure}[!htp]
\centering
\includegraphics[width=16cm,trim=1.5in  0.2in 0.5in 0.75in,clip=true]{chapter6_grey_radtran/Dissertation_Data/S8_vs_S32_Material_Temperature.pdf}
\caption{Comparison of $S_8$ and $S_{32}$ material temperature profiles for Marshak wave problem.}
\label{fig:s8_vs_s32_temperature}
\end{figure}
%
%
\begin{figure}[!htp]
\centering
\includegraphics[width=16cm,trim=1.5in  0.2in 0.5in 0.75in,clip=true]{chapter6_grey_radtran/Dissertation_Data/S8_vs_S32_Radiation.pdf}
\caption{Comparison of $S_8$ and $S_{32}$ angle integrated intensity solutions for Marshak wave problem.}
\label{fig:s8_vs_s32_radiation}
\end{figure}
Even with $S_{32}$ Gauss quadrature, we continue to see non-smooth dips in both the material temperature and angle integrated intensity profiles, however the dips are significantly smaller for the $S_{32}$ solution as compared to the $S_8$ solution, particularly for the material temperature profile.
In \fig{fig:s8_vs_s32_radiation}, we no longer observe one dip in the angle integrated intensity profile, but rather four smaller dips.
Suspecting these are caused by glancing incidence angles in the quadrature set, we plot the angular intensity for all $\mu_d > 0$ in \fig{fig:s32_intensity}.
\begin{figure}[!htp]
\centering
\includegraphics[width=16cm,trim=1.5in  0.2in 0.5in 0.75in,clip=true]{chapter6_grey_radtran/Dissertation_Data/S32_Intensity.pdf}
\caption{$S_{32}$ intensity solutions for all $\mu_d > 0$, for the Marshak wave problem.}
\label{fig:s32_intensity}
\end{figure}
As with the $S_8$ solution, the dips in $\phi$ are associated with corresponding dips in $I_d$ for glancing $\mu_d>0$.\
The discontinuity associated with $\mu_d = 0.3319$ is obscured in \fig{fig:s8_vs_s32_radiation} as the dip occurs just as the $S_{32}$ angle integrated intensity cross over the $S_8$ angle integrated intensity solution.
The $\mu_d=0.0483$ intensity jump in \fig{fig:s32_intensity} causes the greatest effect in \fig{fig:s8_vs_s32_radiation} for two reasons.  
First, the most glancing quadrature angle is attenuated the most rapidly and as such would be expected to have the greatest drop in value.
Second, Gauss angular quadrature assigns the greatest weight to the quadrature points most near $\mu_d = 0$. 
Surprisingly, the $S_8$ and $S_{32}$ calculations have nearly identical positions and values of the temperature and angle integrated intensity solution near the problem boundary and the hot/cold interface.  
If however, the goal is a smooth transport solution, the value of $N$ required to create a smooth $S_N$ $\phi$ solution appears to be much higher than $S_{32}$, due to the presence of ray effects \cite{lewis_book}.

The kinks observed in the higher order $S_N$ solutions for the Marshak wave problem are examples of time ray effects.
Typically these are not observed in thermal radiative transfer simulations, because as the material heats up, the magnitude of photon re-emission sources quickly becomes comparable to and surpasses the incident photon contribution to the angular intensity.
To verify that the observed kinks are indeed time ray effects, consider \fig{fig:phi_time_slices} and \fig{fig:temp_time_slices} which show the $S_{32}$ Marshak wave radiation energy density and material temperature at different points in time, computed using 1000 spatial cells, cubic SLXS Lobatto, $\Delta t_{max} = 2 \times 10^{-4}$, and the 2-2 SDIRK time scheme.
If the observed kinks are time ray effects, they will be worse at earlier times and less pronounced at later times, due to the attenuation of the incident intensity contribution over a greater distance (as the wave front advances) and photon re-emission from the increased material temperature.
\begin{figure}[!htp]
\centering
\includegraphics[width=17cm,trim=2in  0.5in 0.5in 0.75in,clip=true]{chapter6_grey_radtran/Dissertation_Data/S32_Time_Ray_Effects_Radiation_Cv1_SigA1.pdf}
\caption{$S_{32}$ radiation energy density at different times for Marshak wave problem.}
\label{fig:phi_time_slices}
\end{figure}
\begin{figure}[!hbp]
\centering
\includegraphics[width=17cm,trim=2in  0.4in 0.5in 0.75in,clip=true]{chapter6_grey_radtran/Dissertation_Data/S32_Time_Ray_Effects_Temperature_Cv1_SigA1.pdf}
\caption{$S_{32}$ material temperature solution at different times for Marshak wave problem.}
\label{fig:temp_time_slices}
\end{figure}
While the material temperature solution does not have visually large kinks due to ray effects, ray effects can be seen in the radiation energy density solution.
To observe that the radiation energy density ray effects decrease in magnitude at later times, first consider the radiation energy density at $t=0.1$, given in \fig{fig:t01_radiation_energy}, and then compare to the radiation energy density at $t=2.0$ given in \fig{fig:t2_radiation_energy}.  
Both figures use the same $y$-axis scaling.  
However, it is clear that the radiation energy drops associated with ray effects are significantly larger at $t=0.1$ than at $t=2.0$.
\begin{figure}[!htp]
\centering
\includegraphics[width=17cm,trim=2in  0.5in 0.5in 0.75in,clip=true]{chapter6_grey_radtran/Dissertation_Data/S32_T01_Radiation_Equal_Height.pdf}
\caption{$S_{32}$ radiation energy density at $t=0.1$ for Marshak wave problem.}
\label{fig:t01_radiation_energy}
\end{figure}
\begin{figure}[!hbp]
\centering
\includegraphics[width=17cm,trim=2in  0.4in 0.5in 0.75in,clip=true]{chapter6_grey_radtran/Dissertation_Data/S32_T2_Radiation.pdf}
\caption{$S_{32}$ radiation energy density at $t=2$ for Marshak wave problem.}
\label{fig:t2_radiation_energy}
\end{figure}
The $\mu$ labels in \figs{fig:t01_radiation_energy}{fig:t2_radiation_energy} correspond to the directional cosines of angular intensities that experience a significant drop at those locations.
 as shown in \fig{fig:t01_intensity} and \fig{fig:t2_intensity} for $t=0.1$ and $t=2$, respectively.
\begin{figure}[!htp]
\centering
\includegraphics[width=17cm,trim=2in  0.5in 0.5in 0.75in,clip=true]{chapter6_grey_radtran/Dissertation_Data/S32_T01_Intensity.pdf}
\caption{$S_{32}$ angular intensity for $\mu_d>0$ at $t=0.1$ for the Marshak wave problem.}
\label{fig:t01_intensity}
\end{figure}
\begin{figure}[!hbp]
\centering
\includegraphics[width=17cm,trim=2in  0.4in 0.5in 0.75in,clip=true]{chapter6_grey_radtran/Dissertation_Data/S32_T2_Intensity.pdf}
\caption{$S_{32}$ angular intensity for $\mu_d>0$ at $t=2$ for the Marshak wave problem.}
\label{fig:t2_intensity}
\end{figure}
Comparing \fig{fig:t01_intensity} to \fig{fig:t2_intensity} clearly shows that as time progresses, ray effects decrease, in \fig{fig:t01_intensity}, there are six angular intensities that have a discontinuous jump from being dominated by incident boundary conditions and heated material photon re-emission whereas in \fig{fig:t2_intensity}, at most 4 directions experience a non-smooth transition from being dominated by incident boundary contributions ti being dominated by photon re-emission.

Though ray effects are inherent to discrete ordinates calculations, the severe time ray effects observed in this Marshak wave problem are more a function of problem parameters than of a severe fundamental flaw with discrete ordinates methods for radiative transfer.
The choice to define $a=c=C_v=1$ and $\sigma_t = \sigma_a = \frac{1}{T^3}$ was chosen by previous authors on the basis of simplicity alone, not scaling of physical quantities.
This can easily be seen by considering an alternative simulation, where we define $\sigma_t = \sigma_a = \frac{1000}{T^3}$.
Under this assumption, even the heated material is optically thick, and photon re-emission quickly becomes the more dominant contributor to angular intensity than incident photon energy.
In \fig{fig:sig_a_1000_radiation}, radiation energy density solutions for this problem at different times are given for a simulation using 1000 spatial cells, $x\in[0,1]$, linear SLXS Lobatto, and 2-2 time differencing.
\begin{figure}[!htp]
\centering
\includegraphics[width=8cm,trim=1in  0.5in 1in 0.75in,clip=true]{chapter6_grey_radtran/Dissertation_Data/More_Times_P1_S8_Time_Ray_Effects_Radiation_Cv1_SigA1000.pdf}
\caption{$S_{8}$ radiation energy density solutions for modified Marshak wave problem with $\sigma_a = \frac{1000}{T^3}$ at different times.}
\label{fig:sig_a_1000_radiation}
\end{figure}
\begin{figure}[!hbp]
\centering
\includegraphics[width=8cm,trim=1in  0.6in 1.0in 0.75in,clip=true]{chapter6_grey_radtran/Dissertation_Data/More_Times_P1_S8_Time_Ray_Effects_Temperature_Cv1_SigA1000.pdf}
\caption{$S_{8}$ material temperature solutions for modified Marshak wave problem with $\sigma_a = \frac{1000}{T^3}$ at different times.}
\label{fig:sig_a_1000_temperature}
\end{figure}
Figures \ref{fig:sig_a_1000_radiation}-\ref{fig:sig_a_1000_temperature} make it clear that changing $\sigma_a$ fundamentally alters the ``unity'' Marshak wave problem.
The thermal wave does not penetrate nearly as far, but also does not exhibit any time ray effects.

\section{Effectiveness of MIP DSA for TRT Iterative Acceleration}
\label{sec:mip_results}

As of yet, we have failed to have any discussion of the iterative performance of MIP DSA applied to the grey thermal radiative transfer equations.
Though the problems we have considered are not necessarily optically thick, we have used MIP DSA to solve all problems.
A large number of the problems we have considered are not optically thick, in part because we were interested in spatial error convergence.
In \tbl{tbl:iteration_count}, we give a sampling of the average number of iterations required to update the intensity for a given thermal iteration for the problems we have considered thus far.

Several observations can be made regarding the data in \tbl{tbl:iteration_count}.  
Most importantly, MIP DSA applied to the grey TRT is a stable iterative scheme and at worst requires as many iterations as source iteration alone.
Also, the number of iterations for MIP DSA and SI are nearly equal only for most of the problems we have considered, but for those where SI requires larger number of iterations, the ratio of SI+DSA iterations to SI alone iterations grows.
Finally, MIP DSA is compatible with the self-lumping DFEM schemes we have developed that explicitly account for the within cell variation of opacity and heat capacity.

\begin{table}[!htp]
\centering
\caption{Iteration count for different TRT model problems.}
\begin{tabular}{|c|c|c|c|}
\hline
Problem Description & Scheme & Average DSA+SI & Average SI \\
{}									&				 &  Iterations & Iterations  \\
\hline
MMS Constant Time & Linear  & 1.4 & 2.4 \\
4 cells 					& SLXS Lobatto & {} & {} \\
\hline
MMS Constant Time	 & Cubic 			 & 1.6 & 2.3 \\
8 cells 						& SLXS Lobatto & {} & {} \\
\hline
MMS Constant Time	 & Cubic 				 & 1.8 & 1.8 \\
128 cells 					& SLXS Lobatto & {} & {} \\
\hline
MMS1 						& Quadratic 		& 2.0 & 13.5 \\
2 cells 				& SLXS Gauss 		& {}  & {} \\
\hline
MMS1 						& Quadratic 	& 3.0 & 13.6 \\
32 cells 				& SLXS Gauss 	& {} & {} \\
\hline
MMS1	 				& Quadratic  & 4.0 & 13.5 \\
128 cells 		& SLXS Gauss & {} & {} \\
\hline
MMS1 					& Quadratic		& 4.2 & 13.5 \\
256 cells 		& SLXS Gauss 	& {} & {} \\
\hline
MMS2 						& Linear	 & 1.0 & 2.7 \\
2 cells 					& SLXS Gauss & {} & {} \\
\hline
MMS Constant Space 									& Quartic 				 & 17.0 & 39.0 \\
Alexander 3-3, $\Delta t=1$					& SLXS Lobatto 			& {}  & {} \\
\hline
MMS Constant Space 										& Quartic 				 & 6.6 & 11.7 \\
Alexander 3-3, $\Delta t=\frac{1}{8}$	& SLXS Lobatto 			& {}  & {} \\
\hline
MMS Constant Space 										& Quartic 				 & 2.3 & 4.9 \\
Alexander 3-3, $\Delta t=\frac{1}{128}$	& SLXS Lobatto 			& {}  & {} \\
\hline
Marshak Wave 									&  Linear			 & 2.1 & 2.9 \\
20 cells, largest $\Delta t$	& SLXS Lobatto & {} 		 & {} \\
\hline
\end{tabular}
\label{tbl:iteration_count}
\end{table}

To demonstrate the iterative effectiveness of MIP DSA we now present a problem designed solely to be optically thick.
We again define a dimensionless problem, $a=c=1$.
We assume a constant $C_v = 0.05$, define $x\in[0,100]$, $t\in[0,5]$, $\sigma_s = 0$, and $\sigma_a = \frac{5000}{T^2}$.
Initially, the slab is in thermodynamic equilibrium at a temperature of $T=0.5$, and is heated with an incident current of $100$ on the left hand side of the slab.
We difference the problem with linear SLXS Lobatto using 50 spatial cells, implicit Euler in time differencing, and a maximum time step size of $\Delta t_{max} = 0.1$.
The average number of transport iterations per thermal iteration is given in \tbl{tbl:high_iter_count}.
\begin{table}[!ht]
\centering
\caption{Iteration count for a very optically thick TRT problem.}
\label{tbl:high_iter_count}
\begin{tabular}{|c|c|}
\hline
Intensity  						& Average Intensity					\\				
Iterative Strategy		& Iterations Per Thermal Iteration \\
\hline
DSA		&   10  \\ 
\hline
SI  &   18378 \\
\hline
\end{tabular}
\end{table}
Clearly, MIP DSA can significantly reduce the iterative work required to solve the grey TRT equations, but the majority of problems we have considered are not very optically thick.

In optically thick and diffusive problems such as this, the traditional convergence condition of 
\benum
\norm{ \phi^{(\ell+1) } - \phi^{(\ell)} } < \epsilon_{\phi} \pec
\eenum
can lead to false convergence \cite{adams_larsen_fast_iterative}.
Noting that our chosen point-wise convergence condition $\text{change\_phi} < \epsilon_{\phi}$, with $\text{change\_phi}$ as defined in \eqt{eq:change_phi}  is not a true mathematical norm, we would still like to investigate the issue of false convergence.
To do so, we first define a norm based convergence condition:
\benum
\norm{ \phi^{(\ell+1)} - \phi^{(\ell)} }_{L^1} < \epsilon_{\phi} \norm{ \phi^{(\ell+1)}}_{L^1} \pep
\label{eq:l1}
\eenum
In \eqt{eq:l1}, we have multiplied the right hand side by $\norm{ \phi^{(\ell+1)} }$ to allow for a uniform convergence criteria, regardless of the physical scale of any particular problem.
As shown in \cite{adams_larsen_fast_iterative}, we may eliminate false convergence by normalizing our convergence condition with $1-\rho$,
\benum
\frac{\norm{\phi^{(\ell+1)} - \phi^{(\ell)}_{L^1} } }{ 1- \rho}< \epsilon_{\phi} \pec
\label{eq:rho_convergence}
\eenum
where $\rho$ is the spectral radius, which we estimate as:
\benum
\rho \approx \frac{ \norm{\phi^{(\ell+1)} - \phi^{(\ell)} } }{ \norm{\phi^{(\ell)} - \phi^{(\ell-1)} } }\pep
\eenum
In \tbl{tbl:rho_iters}, we plot the average number of iterations required to converge $\phi$ for each time step for our optically thick and diffusive test problem.  For all methods, we used $\epsilon_{\phi} = 10^{-10}$.
\begin{table}[!hbp]
\centering
\caption{Average number of inner iterations per thermal iteration using different convergence criteria.}
\label{tbl:rho_iters}
\begin{tabular}{|c|c|c|c|}
\hline
{}  &  $\text{change\_phi} <\epsilon_{\phi}$ & 
		$ \frac{\norm{ \phi^{(\ell+1)} - \phi^{(\ell)} }_{L^1} }{ \norm{ \phi^{(\ell+1)}}_{L^1}  } < \epsilon_{\phi}$  & 
		$\frac{\norm{\phi^{(\ell+1)} - \phi^{(\ell)}_{L^1} } }{ 1- \rho}< \epsilon_{\phi} $\\
		\hline
SI & 18378 & 6382 & 22563 \\
\hline
DSA & 10 & 7 & 7 \\
\hline
\end{tabular}
\end{table}
Table \ref{tbl:rho_iters} clearly indicates that for optically thick, diffusive problems, the near unity spectral radius can lead to false iterative convergence, and as such should be accounted for explicitly via \eqt{eq:rho_convergence}.  
We also remark that our choice of using $\norm{\phi^{(\ell+1)} }_{L^1}$ as a physical scaling constant could possibly be improved upon, as $\norm{\phi^{(\ell+1)}}_{L^1}$ is dominated by the already heated region, whereas the greatest changes in $\phi$ are occurring near the thermal wavefront.



