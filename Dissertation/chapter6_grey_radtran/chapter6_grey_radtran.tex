%%%%%%%%%%%%%%%%%%%%%%%%%%%%%%%%%%%%%%%%%%%%%%%%%%%
%
%  New template code for TAMU Theses and Dissertations starting Fall 2012.  
%  For more info about this template or the 
%  TAMU LaTeX User's Group, see http://www.howdy.me/.
%
%  Author: Wendy Lynn Turner 
%	 Version 1.0 
%  Last updated 8/5/2012
%
%%%%%%%%%%%%%%%%%%%%%%%%%%%%%%%%%%%%%%%%%%%%%%%%%%%
%%%%%%%%%%%%%%%%%%%%%%%%%%%%%%%%%%%%%%%%%%%%%%%%%%%%%%%%%%%%%%%%%%%%%%
%%                           SECTION III
%%%%%%%%%%%%%%%%%%%%%%%%%%%%%%%%%%%%%%%%%%%%%%%%%%%%%%%%%%%%%%%%%%%%%



\chapter{\uppercase{Grey Thermal Radiative Transfer}}
\label{sec:chapter6_grey_radtran}

We now apply our self-lumping DFEM methodology to the grey thermal radiative transfer equations.
Our framework will share important characteristics of the work presented by Morel, Wareing, and Smith\cite{morel_radtran}.
That is, we will
\begin{enumerate}
\item linearize the Planckian about an arbitrary temperature,
\item  expand the angular intensity and temperature in a $P$ degree trial space, and
\item expand the spatial dependence of the Planckian in a $P$ degree trial space.
\end{enumerate}
There are several differences between the work we present here and that of \cite{morel_radtran}.
First, we will derive our method for arbitrary DFEM polynomial trial space degree, not only a linear polynomial trial space.
Second, \cite{morel_radtran} used a DFEM scheme equivalent to traditional lumping, whereas we will primarily be considering quadrature based self-lumping discretizations.
However, the equations we derive will be applicable to any DFEM scheme that uses polynomial trial and test functions.
Additionally, we will consider arbitrary order (and stage count) SDIRK integration, not only implicit Euler time integration.
The two most important differences between our work here and that of \cite{morel_radtran} will be that we will assume opacity can vary within each spatial cell and we will fully converge the Planckian linearization in temperature.
As shown by Larsen, Kumar, and Morel, failure to fully converge the Planckian linearization in temperature can result lead to non-physical solutions that violate the maximum principle\cite{larsen_trt}.

The remainder of this chapter will be divided into three sections.  Section \ref{sec:chap6_linearization} will describe our linearization and discretization of grey TRT equations.  
In \secref{sec:chap6_programming}, we will briefly describe the features and computer science implementation of the discretized TRT equations.  
Finally, we present numerical results that verify and demonstrate the capabilities of our TRT solver in \secref{sec:chap6_results} and discuss the results in \secref{sec:chap6_conclusions}.

\section{Linearization and Discretization of Grey TRT Equations}
\label{sec:chap6_linearization}
We will begin our discussion of how to linearize the Planckian of the grey thermal radiative transfer in temperature by first briefly outlining SDIRK temporal integration in \secref{sec:sdirk_explained}.
A more complete explanation can be found in \cite{alexander}

\subsection{SDIRK Time Integration}
\label{sec:sdirk_explained}
The coefficients, $a_i$, $b_i$, and $c_i$ that describe any Runge-Kutta time integration are typically given in formatted tables called Butcher Tableaux.
Due to the stiff nature of the TRT equations, we will limit ourselves to SDIRK time integration schemes.  
Depending on the source, SDIRK stands for Single-Diagonally Implicit Runge-Kutta, S-stable Diagonally Implicit Runge-Kutta, or one of many other expansions of the SDIRK acronym, depending on the source's author.
The Butcher Tableaux of an SDIRK  scheme with $N_{stage}$ stages is given in \eqt{eq:butcher}
\benum
\label{eq:butcher}
\begin{array}{c|c|cccc}
\text{Stage}& c_i 	 & a  			&  		&					&	\\
\hline
1						&  c_1   &  a_{11} 	&  0  	&		\dots		&  0 \\
2						&  c_2   &  a_{21}  & a_{22}  & 		0		& \vdots	\\	
i						& c_i    &   a_{i1} &  a_{i2} & \ddots   &	0	\\
N_{stage}     			&  c_{N_{stage}}   &   a_{N_{stage}1} & a_{N_{stage}2} 	& \dots 		& a_{N_{stage}N_{stage} }\\
\hline
{}					&				&		b_1		&		b_2			& \dots 	&   b_{N_{stage}}
\end{array} \pep
\eenum
To illustrate how SDIRK is used to advance time dependent quantities, let us consider a time dependent scalar function, $g(t)$.
Given an initial value at time (or time step) $t^n$, $g_n = g(t^n)$, then $g(t^{n+1})$ is:
\benum
g_{n+1} = g_n + \Delta t \sum_{i=1}^{N_{tstage}}{b_i k_i} \pec
\label{eq:p1}
\eenum
where $\Delta t = t^{n+1} - t^n$, and $k_i$ is defined as:
\benum
k_i = f\left( t_n + c_i \Delta t ~,~g_{n} + \Delta t \sum_{j=1}^i{c_{ij} k_j }\right) \pec
\eenum
and
\benum
f(t,g) = \frac{\partial g}{\partial t} \pep
\eenum
Equation \ref{eq:p1} can also be interpreted as meaning:
\benum
g_i = g_{n} + \Delta t \sum_{j=1}^i{a_{ij} f\left(t_n + \Delta t c_j , g_j\right)} \pec
\label{eq:psi-def}
\eenum
where $g_i$ is the intermediate value of $g$ at the time of stage $i$, $t_i = t^n + \Delta t c_i$.

\subsection{Spatially Analytic Linearization}
We now linearize the spatially analytic, 1-D slab, grey, discrete ordinates TRT equations with SDIRK time integration.  The fully analytic grey TRT are given in \eqts{eq:analytic_grey_trt},
\begin{subequations}
\label{eq:analytic_grey_trt}
\benum
\frac{1}{c} \frac{\partial I}{\partial t} + \mu_d \frac{\partial I}{\partial x} + \sigma_t I= \frac{1}{4\pi}\sigma_s \phi + \sigma_a B + S_I
\label{eq:intensity_eq}
\eenum
\benum
C_v \frac{\partial T}{\partial t} = \sigma_a \left( \phi - 4\pi B \right) + S_T \pep
\label{eq:temperature_eq} 
\eenum
\end{subequations}
In \eqts{eq:analytic_grey_trt}, we have assumed all scattering and material photon emission is isotropic, assume that $I$ is the intensity with directional cosine $\mu_d$ (relative to the $x$-axis),  $S_I$ is a homogenous intensity source in the direction of $\mu_d$, $S_T$ is a homogenous temperature source, and the frequency integrated Planck, $B$ is:
\benum
B(T) = \frac{1}{4\pi} ac T^4\pec
\eenum
with $c$ as the speed of light, and $a$ is the Planck radiation constant.
To use SDIRK to advance $I$ and $T$ in time, we must first define the time derivatives of $I$ and $T$:
\benum
 \frac{\partial I}{\partial t} = c\left[ \frac{1}{4\pi}\sigma_s \phi + \sigma_a B + S_I - \mu_d \frac{\partial I}{\partial x} - \sigma_t I \right]
\label{eq:k_I}
\eenum
and
\benum
\frac{\partial T}{\partial t} = \frac{1}{C_v} \left[ \sigma_a \left( \phi - 4\pi B \right) + S_T \right] \pep
\label{eq:k_T}
\eenum
We evaluate $k_{I,s}$ and $k_{T,s}$, the SDIRK $k$ values for intensity and temperature for stage $s$ as:
\benum
k_{I,s} = c\left[ \frac{1}{4\pi}\sigma_{s}(T_s) \phi_s + \sigma_a(T_s) B(T_s) + S_I(t_s) - \mu_d \frac{\partial I_s}{\partial x} - \sigma_t(T_s) I_s \right]
\label{eq:k_I_stage}
\eenum
and
\benum
k_{T,s} = \frac{1}{C_v(T_s)} \left[ \sigma_a(T_s) \left( \phi_s - 4\pi B(T_s) \right) + S_T(t_s) \right] \pec
\label{eq:k_T_stage}
\eenum
where $\phi_s$, $I_s$, and $T_s$ are the angle integrated intensity, angular intensity, and temperature at stage/time $s$.
With these definitions, we now seek to find $I_1$ and $T_1$ as prescribed by \eqt{eq:psi-def}.
\benum
I_1 = I_n + a_{11} \Delta t k_{I,1} \pec
\eenum
after substituting the definition of \eqt{eq:k_I_stage}
\benum
I_1 = I_n + a_{11} \Delta t c \left[ \frac{1}{4\pi}\sigma_{s} \phi_1 + \sigma_a B+ S_I - \mu_d \frac{\partial I_1}{\partial x} - \sigma_t I_1 \right] 
\pep
\label{eq:i_1_start}
\eenum
Likewise for temperature we have
\benum
T_1 = T_n +\frac{a_{11} \Delta t }{C_v} \left[ \sigma_a \left( \phi_1 - 4\pi B \right) + S_T  \right] \pep
\label{eq:t_1_start}
\eenum
In \eqt{eq:i_1_start} and \eqt{eq:t_1_start}, we have assumed that unless otherwise noted, all material properties and sources are evaluated at $t_s$ and $T_s$.

We now introduce the linearization of the Planckian in temperature. 
For an arbitrary temperature iterate, $T_*$, we approximate $B(T_s)$ as:
\beanum
B(T) &\approx & B(T_*) + \frac{\partial B}{\partial T} \bigg \lvert_{T=T_*} \left(  T - T_* \right) \\
B(T) &\approx & B_* + D_*  \left(  T - T_* \right) \pep
\label{eq:scalar_linear}
\eeanum
If we could remove the dependence on $T_1$ from \eqt{eq:i_1_start}, we could solve \eqt{eq:i_1_start} using the same techniques that have been developed to solve the discrete ordinates neutron transport equation.
To remove the dependence on $T_1$ from \eqt{eq:i_1_start}, we first insert the linearization of \eqt{eq:scalar_linear} into \eqt{eq:t_1_start} and manipulate.
We begin with \eqt{eq:long_t_1}
\benum
T_1 = T_n + \frac{a_{11} \Delta t }{C_v} \left[ \sigma_a \left( \phi_1 - 4\pi \left(  B_* + D_*  \left(  T_1 - T_* \right)  \right) \right) + S_T  \right] \pep
\label{eq:long_t_1}
\eenum

Since we will linearize the spatially discretized TRT equations in the next section, it may seem unnecessary to perform the linearization for the spatially analytic case.
If we could solve the spatially discretized equations efficiently through source iteration alone, it would be redundant to linearize the the spatially analytic and spatially discretized TRT equations.
However, acceleration is essential for efficient solution of the TRT equations due to the non-linear Planckian absorption/re-emission terms in the linearized radiation intensity equation resulting in a situation analogous to a scattering dominated medium in neutron transport.
This will be more clear and we will re-visit this topic, after we have derived the linearized the spatially discretized TRT.

\section{Computer Science Implementation}
\label{sec:chap6_programming}

\section{Numerical Results}
\label{sec:chap6_results}

\section{Conclusions}
\label{sec:chap6_conclusions}


